\documentclass[12pt]{amsart}
%prepared in AMSLaTeX, under LaTeX2e
\addtolength{\oddsidemargin}{-.6in} 
\addtolength{\evensidemargin}{-.6in}
\addtolength{\topmargin}{-.5in}
\addtolength{\textwidth}{1.2in}
\addtolength{\textheight}{1.0in}

\renewcommand{\baselinestretch}{1.05}

\usepackage{verbatim,fancyvrb}
\usepackage{soul}
\usepackage{palatino}
\usepackage[dvipsnames]{xcolor}

\newtheorem*{thm}{Theorem}
\newtheorem*{defn}{Definition}
\newtheorem*{example}{Example}
\newtheorem*{problem}{Problem}
\newtheorem*{remark}{Remark}

\newcommand{\mtt}{\texttt}
\usepackage{alltt,xspace}
\newcommand{\mfile}[1]
{\medskip\begin{quote}\scriptsize \begin{alltt}\input{#1.m}\end{alltt} \normalsize\end{quote}\medskip}

\usepackage[final]{graphicx}
\newcommand{\mfigure}[1]{\includegraphics[height=2.5in,
width=3.5in]{#1.eps}}
\newcommand{\regfigure}[2]{\includegraphics[height=#2in,
keepaspectratio=true]{#1.eps}}
\newcommand{\widefigure}[3]{\includegraphics[height=#2in,
width=#3in]{#1.eps}}

\usepackage{amssymb}

\usepackage[pdftex, colorlinks=true, plainpages=false, linkcolor=black, citecolor=red, urlcolor=red]{hyperref}

% macros
\newcommand{\br}{\mathbf{r}}
\newcommand{\bv}{\mathbf{v}}
\newcommand{\bx}{\mathbf{x}}
\newcommand{\by}{\mathbf{y}}

\newcommand{\CC}{\mathbb{C}}
\newcommand{\RR}{\mathbb{R}}
\newcommand{\ZZ}{\mathbb{Z}}

\newcommand{\eps}{\epsilon}
\newcommand{\grad}{\nabla}
\newcommand{\lam}{\lambda}
\newcommand{\lap}{\triangle}

\newcommand{\ip}[2]{\ensuremath{\left<#1,#2\right>}}

%\renewcommand{\det}{\operatorname{det}}
\newcommand{\onull}{\operatorname{null}}
\newcommand{\rank}{\operatorname{rank}}
\newcommand{\range}{\operatorname{range}}

\newcommand{\Julia}{\textsc{Julia}\xspace}
\newcommand{\Matlab}{\textsc{Matlab}\xspace}
\newcommand{\Octave}{\textsc{Octave}\xspace}
\newcommand{\Python}{\textsc{Python}\xspace}

\newcommand{\prob}[1]{\bigskip\noindent\textbf{#1}\quad }

\newcommand{\chapexers}[2]{\prob{Chapter #1, pages #2, Exercises:}}
\newcommand{\exer}[2]{\prob{Exercise #1}}

\newcommand{\pts}[1]{(\emph{#1 pts}) }
\newcommand{\epart}[1]{\medskip\noindent\textbf{(#1)}\quad }
\newcommand{\ppart}[1]{\,\textbf{(#1)}\quad }

\newcommand*\circled[1]{\tikz[baseline=(char.base)]{
            \node[shape=ellipse,draw,inner sep=2pt] (char) {#1};}}


\begin{document}
\scriptsize \noindent Math 615 NADE (Bueler) \hfill 10 March, 2025
\normalsize

\medskip\bigskip

\Large\centerline{\textbf{Assignment \#6}}
\large
\bigskip

\centerline{\textbf{Due Monday, 31 March 2025, at the start of class}}
\bigskip
\normalsize

\thispagestyle{empty}

\bigskip
Please read sections 5.3--5.9 and 6.1--6.4 from the textbook (R.~J.~LeVeque, \emph{Finite Difference Methods for Ordinary and Partial Diff.~Eqns.}, SIAM Press 2007).

\medskip


\prob{Problem P24.}  For an autonomous ODE system $u'(t) = f(u(t))$, compute the leading term in the local truncation error of the following 2 methods.  For part \textbf{(a)}, please follow the style of Example 5.9.\footnote{For the multistep midpoint rule (5.23), Example 5.9 finds leading-order term $\frac{1}{6} k^2 u'''(t_n)$.}  For part \textbf{(b)} follow Example 5.11, which handles scheme (5.30), the explicit midpoint rule.  That is, for part \textbf{(b)} you should show the simpler fact $\tau^n=O(k^2)$, without finding the leading-order coefficient.

\epart{a} the 2-step BDF method (5.25).

\medskip
\noindent \emph{Hint.}  Expand around $t_{n+1}$, to get $\tau^{n+1}$.

\epart{b} the explicit trapezoid method,
\begin{align*}
U^\ast &= U^n + k f(U^n) \\
U^{n+1} &= U^n + \frac{k}{2} \Big(f(U^n) + f(U^\ast)\Big),
\end{align*}
which is $U^{n+1} = U^n + \frac{k}{2} \Big(f(U^n) + f\big(U^n + k f(U^n)\big)\Big)$ when combined.


\prob{Problem P25.}  \ppart{a}  In preparation for problem \textbf{P26} below, write two solvers

\centerline{\texttt{function [tt,zz] = fe1(f,eta,t0,tf,N)}}

\centerline{\texttt{function [tt,zz] = rk4(f,eta,t0,tf,N)}}

\noindent which implement the forward Euler scheme $U^{n+1}=U^n+kf(t_n,U^n)$ and the classical RK4 scheme (5.33), respectively, to solve the ODE IVP in (5.1) and (5.2).  The first input to these solvers is \texttt{function z = f(t,u)}.\footnote{Please use the \Matlab and \texttt{scipy.integrate} variable order here, namely $f(t,u)$, and not the book order ``$f(u,t)$.''  Numerical libraries, and their black box solvers like \texttt{ode45} or \texttt{scipy.integrate.solve\_ivp}, generally use $f(t,u)$.  I don't know why LeVeque swapped it, but I advise against following him.}  The other inputs are a vector of initial values \texttt{eta} $= u(t_0)$, the initial time \texttt{t0}, the final time \texttt{tf}, and the number of equal-length steps (subintervals) \texttt{N}, respectively.  Note that the time step is $k = (t_f-t_0)/N$.  Each solver outputs the entire trajectory, so \texttt{tt} is a 1D array of length $N+1$ starting with $t_0$ and ending with $t_f$.\footnote{Again, this is compatible with how the \Matlab and \texttt{scipy.integrate} solvers do things.}  If $\eta\in\RR^s$ then \texttt{zz} is a 2D array with $s$ rows and $N+1$ columns; each column $i$ gives the solution $u(t)$ at the $i$th time in \texttt{tt}.

\epart{b}  Solve the following problem exactly:
    $$x'' + 4 x = 0, \qquad x(0)=1, \quad x'(0)=0.$$
Also write this as a first order system, needed when setting-up numerical solvers.

\epart{c} The problem in \textbf{(b)}, for example on the interval $[t_0,t_f] = [0,5]$, makes a good test case.  Demonstrate that the final-time, absolute numerical error of each solver in \textbf{(a)} converges at the expected rate as $k\to 0$.

\medskip
\noindent \emph{Hint.} What is the expected rate is for each method?  There is no need to compute local truncation errors yourself, but you must know their orders.


\prob{Problem P26.}  \emph{This is a real application.  Perhaps it will help you appreciate our abstract notation for ODE systems, vector data types in our languages, and higher-order explicit ODE schemes.  This problem has an exact solution,\footnote{See, for example: {\scriptsize\url{https://www.diva-portal.org/smash/get/diva2:630427/FULLTEXT01.pdf}}} but it is not used here.}

\medskip
\noindent Consider the problem of two massive bodies (particles) with masses $m_1$ and $m_2$.  They are attracted by gravity only.  They travel in a plane so their positions are given by vector-valued functions $\bx_i(t) = (x_i(t),y_i(t))$ for $i=1,2$.  Newton's second law and Newton's law of gravity combine to say:
\begin{align}
m_1 \bx_1'' &= - G m_1 m_2 \frac{\bx_1 - \bx_2}{|\bx_1 - \bx_2|^3} \label{twobody} \\
m_2 \bx_2'' &= - G m_1 m_2 \frac{\bx_2 - \bx_1}{|\bx_1 - \bx_2|^3} \notag
\end{align}

We will consider the Earth and the Moon in isolation as our example.  Thus the constants are $m_1 = 5.972 \times 10^{24} \,\text{kg}$, $m_2 = 7.348 \times 10^{22} \,\text{kg}$, and $G = 6.674 \times 10^{-11}\,\text{m}^3\,\text{kg}^{-1}\,\text{s}^{-2}$.  We measure $t$ in seconds---convert all times to seconds inside the solver!---and $x_i,y_i$ in meters.  (\emph{Please confirm that the units are consistent in system} (1).)

\epart{a} By using notation $v_i=x_i', w_i=y_i'$ for $i=1,2$, write system (1) as a first-order ODE system of dimension $s=8$, with solution column vector $u(t)\in\RR^8$.  Use the component ordering
\begin{align*}
u(t) &= \begin{bmatrix} x_1(t) & y_1(t) & x_2(t) & y_2(t) & v_1(t) & w_1(t) & v_2(t) & w_2(t) \end{bmatrix}^\top \\
     &= \begin{bmatrix} u_1(t) & u_2(t) & u_3(t) & u_4(t) & u_5(t) & u_6(t) & u_7(t) & u_8(t) \end{bmatrix}^\top.
\end{align*}
That is, write system \eqref{twobody} in the form of (5.1) in the book:\footnote{In fact the right side of this ODE system does not have explicit dependence on $t$, but, to avoid confusion in the implementation, use the \Matlab/\texttt{scipy.integrate} variable ordering.} $u'(t) = f(t,u(t))$.  Then implement a single function

\centerline{\texttt{function z = fearthmoon(t,u)}}

\noindent which computes the right-hand-side function $f(t,u)$ of the ODE system.

\epart{b}  Here are some initial conditions which are vaguely like what they are in reality,\footnote{I searched ``earth moon distance meters'' and ``mean orbital velocity moon.''} at least if you turned off all the gravity of other bodies and start the Earth at the origin: $t_0=0$, $x_1(0)=0,y_1(0)=0,v_1(0)=0,w_1(0)=0$, $x_2(0)=3.844\times 10^8$ meters, $y_2(0)=0$, $v_2(0)=0$, $w_2(0)=1.022\times 10^3 \,\text{m}\,\text{s}^{-1}$.  Use these initial conditions to generate approximate solutions with $t_f=35$ days.

Now use each of the solvers from problem \textbf{P25} with $N=40$ and $N=960$, i.e.~daily and hourly time steps, respectively.  Also use \texttt{ode45()}, or comparable dual-order, adaptive Runge-Kutta black-box solver, using the default accuracy.  That is, generate five numerical solutions.

Do not, of course, show me lots of numbers.  Make basic plots of the computed trajectories, i.e.~the $x_i,y_i$ values.  Describe in a few words what you see, and how these results relate to the local truncation error of the schemes in \textbf{P25}.

\epart{c}  How long in days is a lunar month, using your best computation from part \textbf{(b)}?  


\prob{Problem P27.}  \ppart{a}  For ODE systems of form $u'(t)=Au(t)+g(t)$, where $A$ is a square matrix and $g:\RR\to\RR^s$, build a BDF2 multistep solver for the initial-value problem:

\centerline{\texttt{function [tt,zz] = bdf2(A,g,eta,t0,tf,N)}}

\noindent See equation (5.25).  Address the necessary linear algebra by using the default black box in your language, e.g.~backslash in Matlab, or by pre-factoring.\footnote{Either will get full credit, but the latter should be faster for large $s$.}  Also, choose a scheme for taking the first step which preserves the order of the method.

\epart{b}  Solve exactly the same problem as in \textbf{P25(b)}, as a test case.  Demonstrate that the final-time numerical error converges at the expected rate as $k\to 0$.

\end{document}
