\documentclass[12pt]{amsart}
%prepared in AMSLaTeX, under LaTeX2e
\addtolength{\oddsidemargin}{-.6in} 
\addtolength{\evensidemargin}{-.6in}
\addtolength{\topmargin}{-.5in}
\addtolength{\textwidth}{1.2in}
\addtolength{\textheight}{1.0in}

\renewcommand{\baselinestretch}{1.05}

\usepackage{verbatim,fancyvrb}
\usepackage{soul}
\usepackage{palatino}

\newtheorem*{thm}{Theorem}
\newtheorem*{defn}{Definition}
\newtheorem*{example}{Example}
\newtheorem*{problem}{Problem}
\newtheorem*{remark}{Remark}

\newcommand{\mtt}{\texttt}
\usepackage{alltt,xspace}
\newcommand{\mfile}[1]
{\medskip\begin{quote}\scriptsize \begin{alltt}\input{#1.m}\end{alltt} \normalsize\end{quote}\medskip}

\usepackage[final]{graphicx}
\newcommand{\mfigure}[1]{\includegraphics[height=2.5in,
width=3.5in]{#1.eps}}
\newcommand{\regfigure}[2]{\includegraphics[height=#2in,
keepaspectratio=true]{#1.eps}}
\newcommand{\widefigure}[3]{\includegraphics[height=#2in,
width=#3in]{#1.eps}}

\usepackage{amssymb}

\usepackage[pdftex, colorlinks=true, plainpages=false, linkcolor=black, citecolor=red, urlcolor=red]{hyperref}

% macros
\newcommand{\br}{\mathbf{r}}
\newcommand{\bv}{\mathbf{v}}
\newcommand{\bx}{\mathbf{x}}
\newcommand{\by}{\mathbf{y}}

\newcommand{\CC}{\mathbb{C}}
\newcommand{\RR}{\mathbb{R}}
\newcommand{\ZZ}{\mathbb{Z}}

\newcommand{\eps}{\epsilon}
\newcommand{\grad}{\nabla}
\newcommand{\lam}{\lambda}
\newcommand{\lap}{\triangle}

\newcommand{\ip}[2]{\ensuremath{\left<#1,#2\right>}}

%\renewcommand{\det}{\operatorname{det}}
\newcommand{\onull}{\operatorname{null}}
\newcommand{\rank}{\operatorname{rank}}
\newcommand{\range}{\operatorname{range}}

\newcommand{\Julia}{\textsc{Julia}\xspace}
\newcommand{\Matlab}{\textsc{Matlab}\xspace}
\newcommand{\Octave}{\textsc{Octave}\xspace}
\newcommand{\Python}{\textsc{Python}\xspace}

\newcommand{\prob}[1]{\bigskip\noindent\textbf{#1}\quad }

\newcommand{\chapexers}[2]{\prob{Chapter #1, pages #2, Exercises:}}
\newcommand{\exer}[2]{\prob{Exercise #1}}

\newcommand{\pts}[1]{(\emph{#1 pts}) }
\newcommand{\epart}[1]{\medskip\noindent\textbf{(#1)}\quad }
\newcommand{\ppart}[1]{\,\textbf{(#1)}\quad }

\newcommand*\circled[1]{\tikz[baseline=(char.base)]{
            \node[shape=ellipse,draw,inner sep=2pt] (char) {#1};}}


\begin{document}
\scriptsize \noindent Math 615 NADE (Bueler) \hfill 31 January, 2025
\normalsize

\medskip\bigskip

\Large\centerline{\textbf{Assignment \#3}}
\large
\bigskip

\centerline{\textbf{Due Monday, 10 February 2025, at the start of class}}
\bigskip
\normalsize

\thispagestyle{empty}

\bigskip
Please read sections 2.4--2.16 from the textbook.\footnote{R.~J.~LeVeque, \emph{Finite Difference Methods for Ordinary and Partial Diff.~Eqns.}, SIAM Press 2007}


\prob{Problem P12.}  \ppart{a}  Set up a finite difference method for the general linear, second-order, Dirichlet, two-point boundary value problem:
\begin{equation}
u''(x) + p(x)\, u'(x) + q(x)\, u(x) = f(x), \quad u(a) = \alpha, \quad u(b) = \beta. \label{genbvp}
\end{equation}
Here $p(x),q(x),f(x)$ are all arbitrary functions.  Use approximation (1.3), namely the centered finite difference $D_0$, for the $u'$ term, and the usual centered formula $D^2$ for the $u''$ term.  State the $A$ and $F$, which appear in your linear system $AU=F$, assuming $m$ unknowns $U_1,\dots,U_m$, on an equally-spaced grid with $m$ interior points.

\epart{b} Implement this method.  A recommended approach is to modify an existing, tested solver, for example:

\centerline{\href{https://bueler.github.io/nade/assets/codes/S25/second.m}{\texttt{bueler.github.io/nade/assets/codes/S25/second.m}}}

\medskip
\noindent The recommended Matlab form of your new solver is a function with signature something like

\smallskip
\small
\centerline{\texttt{function [x, U] = generalbvp(p, q, f, a, b, alpha, beta, m)}}

\normalsize
\medskip
\noindent Turn in a listing of the code itself, but please do part \textbf{(c)} to test and debug it before posting your final version.

\epart{c}  Use the method of manufactured solutions to debug and test your code, especially to verify that you are handling $p(x)$ and $q(x)$ correctly.  To avoid making things \emph{too} complicated, fix the simple values $a=0,b=1,\alpha=0,\beta=0$ for all testing.  Generate a first exact solution in the case where $q(x)=0$ but $p(x)$ is non-zero and non-constant.  Then generate a second exact solution in the case where $p(x)=0$ but $q(x)$ is non-zero and non-constant.  (\emph{In these exact solutions you will \emph{choose} $u(x)$ as a function which satisfies the boundary conditions, and \emph{choose} $p(x)$ or $q(x)$, whichever is non-zero, and then you will differentiate and simplify to get $f(x)$ from these assumptions.  You will need to make simple choices of $u(x),p(x),q(x)$, but not \emph{too} simple.})  Now compute error norms for each of the two cases, i.e.~$\|E^h\|_2=\|U^h-\hat U^h\|_2$, and produce a convergence plot showing the expected $O(h^2)$ convergence rate.


\prob{Problem P13.}  \emph{Sometimes these Dirichlet boundary value problems are ill-posed!}

\smallskip
\noindent Consider the following linear BVP with Dirichlet boundary conditions:
\begin{equation}
u''(x) +  u(x) = 0 \quad \text{for $a< x< b$}, \qquad u(a)=\alpha, \qquad u(b)=\beta.  \label{dirbvp}
\end{equation}

\epart{a} Determine the exact solution to BVP \eqref{dirbvp} when $a = 0, b = 1, \alpha = 2, \beta = 3$.  (\emph{Observe that there is exactly one solution for these parameters!})  Test your code from \textbf{P12}, using this exact solution, and demonstrate convergence at the expected $O(h^2)$ rate.  

\epart{b} Instead let $a=0$ and $b=\pi$ in BVP \eqref{dirbvp}.  For what values of $\alpha$ and $\beta$ does \eqref{dirbvp} have any solutions?  First identify a case where there is no solution.  Second, identify a case where there are infinitely-many solutions, and sketch that family of solutions.


\prob{Problem P14.}  Section 2.13 addresses the problem
\begin{equation}
u''(x) = f(x), \qquad u'(0)=\sigma_0, \qquad u'(1)=\sigma_1,  \label{neubvp}
\end{equation}
which has Neumann conditions at both ends.  In the steady heat equation interpretation of BVP \eqref{neubvp}, the values $\sigma_0$ and $\sigma_1$ are heat fluxes.  (\emph{The Dirichlet condition values $\alpha,\beta$ in BVPs \eqref{genbvp} and \eqref{dirbvp} are temperatures.})

\epart{a} Apply the ``second approach'' from Section 2.12 to BVP \eqref{neubvp}.  Show that you get linear system (2.58) as shown in the text.

\epart{b} Generate $A$, the matrix from (2.58), in Matlab/etc; choose $m=10$ here for example.  (\emph{Modify existing code to do this assembly?  In that case show me your code.})  Enter the constant vector $e=[1,1,\dots,1]^\top$, and confirm numerically that $e$ is in the null space of $A$.

\epart{c} The text in Section 2.13 explains a condition that $f(x)$ and $\sigma_0$ and $\sigma_1$ must satisfy so that there \emph{is} a solution to BVP \eqref{neubvp}.  If that condition is satisfied then there are actually infinitely many solutions.  Choose such a case and see what happens numerically, that is, when you solve linear system (2.58) in Matlab/etc.  Explain.
\end{document}
