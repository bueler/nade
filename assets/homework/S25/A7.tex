\documentclass[12pt]{amsart}
%prepared in AMSLaTeX, under LaTeX2e
\addtolength{\oddsidemargin}{-.6in} 
\addtolength{\evensidemargin}{-.6in}
\addtolength{\topmargin}{-.5in}
\addtolength{\textwidth}{1.2in}
\addtolength{\textheight}{1.0in}

\renewcommand{\baselinestretch}{1.05}

\usepackage{verbatim,fancyvrb}
\usepackage{soul}
\usepackage{palatino}
\usepackage[dvipsnames]{xcolor}

\newtheorem*{thm}{Theorem}
\newtheorem*{defn}{Definition}
\newtheorem*{example}{Example}
\newtheorem*{problem}{Problem}
\newtheorem*{remark}{Remark}

\newcommand{\mtt}{\texttt}
\usepackage{alltt,xspace}
\newcommand{\mfile}[1]
{\medskip\begin{quote}\scriptsize \begin{alltt}\input{#1.m}\end{alltt} \normalsize\end{quote}\medskip}

\usepackage[final]{graphicx}
\newcommand{\mfigure}[1]{\includegraphics[height=2.5in,
width=3.5in]{#1.eps}}
\newcommand{\regfigure}[2]{\includegraphics[height=#2in,
keepaspectratio=true]{#1.eps}}
\newcommand{\widefigure}[3]{\includegraphics[height=#2in,
width=#3in]{#1.eps}}

\usepackage{amssymb}

\usepackage[pdftex, colorlinks=true, plainpages=false, linkcolor=black, citecolor=red, urlcolor=red]{hyperref}

% macros
\newcommand{\br}{\mathbf{r}}
\newcommand{\bv}{\mathbf{v}}
\newcommand{\bx}{\mathbf{x}}
\newcommand{\by}{\mathbf{y}}

\newcommand{\CC}{\mathbb{C}}
\newcommand{\RR}{\mathbb{R}}
\newcommand{\ZZ}{\mathbb{Z}}

\newcommand{\eps}{\epsilon}
\newcommand{\grad}{\nabla}
\newcommand{\lam}{\lambda}
\newcommand{\lap}{\triangle}

\newcommand{\ip}[2]{\ensuremath{\left<#1,#2\right>}}

%\renewcommand{\det}{\operatorname{det}}
\newcommand{\onull}{\operatorname{null}}
\newcommand{\rank}{\operatorname{rank}}
\newcommand{\range}{\operatorname{range}}

\newcommand{\Julia}{\textsc{Julia}\xspace}
\newcommand{\Matlab}{\textsc{Matlab}\xspace}
\newcommand{\Octave}{\textsc{Octave}\xspace}
\newcommand{\Python}{\textsc{Python}\xspace}

\newcommand{\prob}[1]{\bigskip\noindent\textbf{#1}\quad }

\newcommand{\chapexers}[2]{\prob{Chapter #1, pages #2, Exercises:}}
\newcommand{\exer}[2]{\prob{Exercise #1}}

\newcommand{\pts}[1]{(\emph{#1 pts}) }
\newcommand{\epart}[1]{\medskip\noindent\textbf{(#1)}\quad }
\newcommand{\ppart}[1]{\,\textbf{(#1)}\quad }

\newcommand*\circled[1]{\tikz[baseline=(char.base)]{
            \node[shape=ellipse,draw,inner sep=2pt] (char) {#1};}}


\begin{document}
\scriptsize \noindent Math 615 NADE (Bueler) \hfill 2 April, 2025
\normalsize

\medskip\bigskip

\Large\centerline{\textbf{Assignment \#7}}
\large
\bigskip

\centerline{\textbf{Due Friday, 11 April 2025, at the start of class}}
\bigskip
\normalsize

\thispagestyle{empty}

\bigskip
Please read Chapters 6, 7, and 8 of the textbook (R.~J.~LeVeque, \emph{Finite Difference Methods for Ordinary and Partial Diff.~Eqns.}, SIAM Press 2007).  Within this material we are de-emphasizing multistep methods, so sections 5.9, 6.4, 7.3, 7.6.1, and 7.7 are all optional.  However, full understanding of the other sections is expected; please actually set aside a bit of time and \emph{read}!

\medskip


\prob{Problem P28.}  Please reproduce Table 7.1.  That is, consider the scalar ODE IVP
	$$u'(t) = \lambda \left( u(t) - \cos(t) \right) - \sin(t), \qquad u(0)=1$$
Use $\lambda=-2100$.  Apply forward Euler\footnote{Re-using an old code is just fine, but of course please pay attention to the details!} to compute approximations of $u(T)$ for $T=2$, for the given values of $k$, and report the final-time numerical errors $|U^N - u(T)|$ as in the Table.  Confirm by this experiment that there is a critical value of $k$ around 0.00095 where the error finally drops from enormous values to something comparable to, and then much smaller than, the solution magnitude itself.

\medskip
\noindent \emph{The book's explains why: $|1+k\lambda| \le 1$ only if $k|\lambda|<2$ or equivalently $k < 2/|\lambda| = 2/2100 = 0.00095238$.  There is no need to include this analysis in your answer; please just confirm it experimentally.}


\prob{Problem P29.}  Consider $\theta$-\emph{methods} for $u' = f(t,u)$:
   $$U^{n+1} = U^n + k\Big[(1-\theta)f(t_n,U^n) + \theta f(t_{n+1},U^{n+1})\Big]$$
Here $0\le \theta \le 1$ is a fixed parameter.

\epart{a} Cases $\theta = 0,1/2,1$ are all familiar methods.  Name them.  Then find the exact absolute stability regions for $\theta = 0,1/4,1/2,3/4,1$.  (\emph{Hint. Apply method to the test equation, and simplify.  Write the complex number $z=k\lambda$ as $z=x+iy$.  Find discs!})

\epart{b} Show they are A-stable for any $\theta \geq 1/2$.


\prob{Problem P30.}  \ppart{a}  For the classical RK4 method, which is Example 5.13 in the textbook, show that the stability function is $R(z) = 1 + z + \frac{1}{2} z^2 + \frac{1}{3!} z^3 + \frac{1}{4!} z^4$.  (\emph{Hint.  Apply to the test equation.  I skipped details for this during lecture; please fill them in.})

\epart{b}  Use a filled-contour plotter, like Matlab's \texttt{contourf} as shown in class, to plot the region of absolute stability of RK4.  (\emph{Hint.  I did this in lecture for an RK2 method.})


\prob{Problem P31.}  Consider this Runge-Kutta method, an \emph{implicit} and one-step interpretation of the midpoint method:
\begin{align*}
U^* &= U^n + \frac{k}{2} f\big(t_n + k/2, U^*\big),\\
U^{n+1} &= U^n + k f\big(t_n + k/2, U^*\big).
\end{align*}
The first stage uses backward Euler to (implicitly) compute a value at the midpoint.  The second stage is a midpoint method using this value.\footnote{One may show that this scheme has LTE $\tau^n = O(k^2)$, but here there is no request to do so.}  Please determine the region of absolute stability for this (combined) method; please do this exactly!  Is this method A-stable?  Is it L-stable?


\prob{Problem P32.}  For a famously stiff problem, consider the heat PDE
\begin{equation}
u_t = u_{xx}  \label{heat}
\end{equation}
Here $u(t,x)$ might be the temperature in a rod of length one ($0 \le x \le 1$).  Let us set boundary temperatures to zero ($u(t,0)=u(t,1)=0$), and assume some initial temperature distribution $u(0,x)=\eta(x)$.

Suppose we apply the \emph{method of lines} (MOL) to \eqref{heat}.  That is, we discretize the spatial derivatives using the notation from Chapter 2.  Specifically, let us use $m+1$ subintervals, $h=1/(m+1)$, and $x_j = j h$ for $j=0,1,2,\dots,m+1$.  Now $U_j(t) \approx u(t,x_j)$.  By eliminating unknowns $U_0=0$ and $U_{m+1}=0$, and keeping the time derivatives as ordinary derivatives, from \eqref{heat} we get a linear ODE system of dimension $m$,
\begin{equation}
U(t)' = A_m U(t)  \label{mol}
\end{equation}
where $U(t) \in \RR^m$ and $A=A_m$ is \emph{exactly} the matrix in the textbook's equation (2.10).  Note that $U(0)_j = \eta(x_j)$ from the initial condition.

The eigenvalues of $A_m$ are given by equation (2.23) in the textbook:
	$$\lambda_p = \frac{2}{h^2} \left(\cos(p\pi h) - 1\right),$$
for $p=1,\dots,m$.

\epart{a} Please explain why all eigenvalues $\lambda_p$ are points on the negative real axis.  Then justify the following approximation of the largest-magnitude (and most negative) eigenvalue:
	$$\lambda_m \approx - 4 (m+1)^2.$$
(\emph{Hint.  Use a Taylor expansion of $\cos(\theta)$ around the right location.})

\epart{b}  Suppose forward Euler is applied to solve \eqref{mol} with equal time steps $k>0$.  How small must $k$ be chosen so that all of the values $z_p=\lambda_p k$, for $p=1,\dots,m$, are inside the region of absolute stability of forward Euler?  Your answer will depend on $m$, but not $p$.  You may use the approximation in part \textbf{(a)} for your analysis.

\epart{c}  What happens to the maximum stable time step for forward Euler, i.e.~the answer from \textbf{b}, when you double the spatial grid resolution?  With the same doubling, what happens to the cost of solving the heat equation problem out to some time $T>0$?

\end{document}
