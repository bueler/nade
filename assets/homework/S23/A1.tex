\documentclass[12pt]{amsart}
%prepared in AMSLaTeX, under LaTeX2e
\addtolength{\oddsidemargin}{-.6in} 
\addtolength{\evensidemargin}{-.6in}
\addtolength{\topmargin}{-.5in}
\addtolength{\textwidth}{1.2in}
\addtolength{\textheight}{1.0in}

\renewcommand{\baselinestretch}{1.05}

\usepackage{verbatim,fancyvrb}
\usepackage{soul}
\usepackage{palatino}

\newtheorem*{thm}{Theorem}
\newtheorem*{defn}{Definition}
\newtheorem*{example}{Example}
\newtheorem*{problem}{Problem}
\newtheorem*{remark}{Remark}

\newcommand{\mtt}{\texttt}
\usepackage{alltt,xspace}
\newcommand{\mfile}[1]
{\medskip\begin{quote}\scriptsize \begin{alltt}\input{#1.m}\end{alltt} \normalsize\end{quote}\medskip}

\usepackage[final]{graphicx}
\newcommand{\mfigure}[1]{\includegraphics[height=2.5in,
width=3.5in]{#1.eps}}
\newcommand{\regfigure}[2]{\includegraphics[height=#2in,
keepaspectratio=true]{#1.eps}}
\newcommand{\widefigure}[3]{\includegraphics[height=#2in,
width=#3in]{#1.eps}}

\usepackage{amssymb}

\usepackage[pdftex, colorlinks=true, plainpages=false, linkcolor=black, citecolor=red, urlcolor=red]{hyperref}

% macros
\newcommand{\br}{\mathbf{r}}
\newcommand{\bv}{\mathbf{v}}
\newcommand{\bx}{\mathbf{x}}
\newcommand{\by}{\mathbf{y}}

\newcommand{\CC}{\mathbb{C}}
\newcommand{\RR}{\mathbb{R}}
\newcommand{\ZZ}{\mathbb{Z}}

\newcommand{\eps}{\epsilon}
\newcommand{\grad}{\nabla}
\newcommand{\lam}{\lambda}
\newcommand{\lap}{\triangle}

\newcommand{\ip}[2]{\ensuremath{\left<#1,#2\right>}}

%\renewcommand{\det}{\operatorname{det}}
\newcommand{\onull}{\operatorname{null}}
\newcommand{\rank}{\operatorname{rank}}
\newcommand{\range}{\operatorname{range}}

\newcommand{\Julia}{\textsc{Julia}\xspace}
\newcommand{\Matlab}{\textsc{Matlab}\xspace}
\newcommand{\Octave}{\textsc{Octave}\xspace}
\newcommand{\Python}{\textsc{Python}\xspace}

\newcommand{\prob}[1]{\bigskip\noindent\textbf{#1}\quad }

\newcommand{\chapexers}[2]{\prob{Chapter #1, pages #2, Exercises:}}
\newcommand{\exer}[2]{\prob{Exercise #1}}

\newcommand{\pts}[1]{(\emph{#1 pts}) }
\newcommand{\epart}[1]{\medskip\noindent\textbf{(#1)}\quad }
\newcommand{\ppart}[1]{\,\textbf{(#1)}\quad }

\newcommand*\circled[1]{\tikz[baseline=(char.base)]{
            \node[shape=ellipse,draw,inner sep=2pt] (char) {#1};}}


\begin{document}
\scriptsize \noindent Math 615 NADE (Bueler) \hfill 17 January, 2023
\normalsize

\medskip\bigskip

\Large\centerline{\textbf{Assignment \#1}}
\large
\bigskip

\centerline{\textbf{Due Wednesday, 25 January 2023, at the start of class}}
\bigskip
\normalsize

\thispagestyle{empty}

\bigskip
Please read sections 1.1--1.4 and 2.1--2.4 from the textbook.\footnote{R.~J.~LeVeque, \emph{Finite Difference Methods for Ordinary and Partial Diff.~Eqns.}, SIAM Press 2007}  The Problems on this assignment are designed to encourage review of certain important prerequisite topics.  In fact, please find three prerequisite textbooks:
\begin{itemize}
\item Find a \textbf{calculus book}.
\item Find an introductory textbook on \textbf{ordinary differential equations} (ODEs).
\item Find an introductory textbook on \textbf{linear algebra}.
\end{itemize}
You will need these references throughout the semester.  In particular, for this assignment, please review these two topics:
\begin{itemize}
  \item[] Calculus book: Taylor's theorem with the remainder formula.\footnote{Taylor's theorem may be best explained by an undergraduate \textbf{numerical analysis} textbook.}
  \item[] ODEs book: The solution of linear homogeneous constant-coefficient ODEs.
\end{itemize}

\medskip
\prob{Problem P1.}  Calculate $257^{\,1/8}$ to within $10^{-5}$ of the exact value \emph{without} any computing machinery except a pencil or pen.    Prove that your answer has this accuracy.  (\emph{You should use a computer to \emph{check} your by-hand value!  Hint: Taylor on $f(x)=x^{1/8}$.})

\prob{Problem P2.}  Assume $f'$ is continuous.  Derive the remainder formula
\begin{equation}\label{lhrule}
\int_0^a f(x)\,dx = a f(0) + \frac{1}{2} a^2 f'(\nu)
\end{equation}
for some (unknown) $\nu$ between zero and $a$.  (\emph{Hint:  Start by showing $f(x)=f(0)+f'(\xi)x$ where $\xi=\xi(x)$ is some number between $0$ and $x$.})  Use two sentences to explain the meaning of \eqref{lhrule} as an approximation to the integral.  That is, answer the question ``What properties of $f(x)$ or $a$ make the left-endpoint rule $\int_0^a f(x)\,dx \approx a f(0)$ more inaccurate?''

\prob{Problem P3.}  Get started in the programming language of your choice.\footnote{Recommended: \Matlab/\Octave or \Python or \Julia.}  Now work at the command line to compute a finite sum approximation to
	$$\sum_{n=1}^\infty \frac{\sin n}{n^3+1}.$$
Compute the partial (finite) sums for $N=10$ and $N=100$ terms.  Turn your command-line work into a function \texttt{mysum(N)}, defined in a file  \texttt{mysum.m}, and check that it works.  Turn in both the command line session and the code.  (\emph{Hint:  These can be very brief.})  Speaking informally, how close do you think the $N=100$ partial sum is to the infinite sum?

\prob{Problem P4.}  Solve, by hand,
\begin{equation}\label{ODE}
y'' + y' - 6 y = 0, \quad y(2)=0, \quad y'(2)=-1,
\end{equation}
for the solution $y(t)$.  Then find $y(4)$.  Give a reasonable by-hand sketch on $t,y$ axes which shows the initial values, the solution, and the value $y(4)$.

\emph{Note you have made a \emph{prediction} of $y(t)$ at $t=4$, given initial data at $t=2$ and a precise ``law'' about how $y(t)$ evolves in time, namely the differential equation itself.}

\prob{Problem P5.}  Using Euler's method for approximately solving ODEs, write your own program to solve initial value problem \eqref{ODE} to find $y(4)$.  A first step is to convert the second-order ODE into a system of two first-order ODEs.  Use a few different step sizes, decreasing as needed, so that you get apparent three-digit accuracy.  (\emph{Hint:  You may use a black-box ODE solver to \emph{check} your work, but this is not required.})
%Matlab soln:
%  f = @(t,w) [w(2); 6*w(1) - w(2)]
%  [t,v] = ode45(f,[2 4],[0 -1]');
%  plot(t,v), legend("y(t)","y'(t)")
%  hold on, plot(4,(1/5)*(exp(-6) - exp(4)),'o'), hold off

\prob{Problem P6.}  Solve, by hand, the ODE boundary value problem
\begin{equation}\label{ODEBVP}
y'' + 2 y' - 3 y = 0, \quad y(0)=\alpha, \quad y(\tau)=\beta,
\end{equation}
for the solution $y(t)$.  Note that $\alpha,\beta,\tau$ are the data of the problem, so the solution will have these parameters in it.


\end{document}
