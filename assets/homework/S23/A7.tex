\documentclass[12pt]{amsart}
%prepared in AMSLaTeX, under LaTeX2e
\addtolength{\oddsidemargin}{-.6in} 
\addtolength{\evensidemargin}{-.6in}
\addtolength{\topmargin}{-.5in}
\addtolength{\textwidth}{1.2in}
\addtolength{\textheight}{1.0in}

\renewcommand{\baselinestretch}{1.05}

\usepackage{verbatim,fancyvrb}
\usepackage{soul}
\usepackage{palatino}

\newtheorem*{thm}{Theorem}
\newtheorem*{defn}{Definition}
\newtheorem*{example}{Example}
\newtheorem*{problem}{Problem}
\newtheorem*{remark}{Remark}

\newcommand{\mtt}{\texttt}
\usepackage{alltt,xspace}
\newcommand{\mfile}[1]
{\medskip\begin{quote}\scriptsize \begin{alltt}\input{#1.m}\end{alltt} \normalsize\end{quote}\medskip}

\usepackage[final]{graphicx}
\newcommand{\mfigure}[1]{\includegraphics[height=2.5in,
width=3.5in]{#1.eps}}
\newcommand{\regfigure}[2]{\includegraphics[height=#2in,
keepaspectratio=true]{#1.eps}}
\newcommand{\widefigure}[3]{\includegraphics[height=#2in,
width=#3in]{#1.eps}}

\usepackage{amssymb}

\usepackage[pdftex, colorlinks=true, plainpages=false, linkcolor=black, citecolor=red, urlcolor=red]{hyperref}

% macros
\newcommand{\br}{\mathbf{r}}
\newcommand{\bv}{\mathbf{v}}
\newcommand{\bx}{\mathbf{x}}
\newcommand{\by}{\mathbf{y}}

\newcommand{\CC}{\mathbb{C}}
\newcommand{\RR}{\mathbb{R}}
\newcommand{\ZZ}{\mathbb{Z}}

\newcommand{\eps}{\epsilon}
\newcommand{\grad}{\nabla}
\newcommand{\lam}{\lambda}
\newcommand{\lap}{\triangle}

\newcommand{\ip}[2]{\ensuremath{\left<#1,#2\right>}}

%\renewcommand{\det}{\operatorname{det}}
\newcommand{\onull}{\operatorname{null}}
\newcommand{\rank}{\operatorname{rank}}
\newcommand{\range}{\operatorname{range}}

\newcommand{\Julia}{\textsc{Julia}\xspace}
\newcommand{\Matlab}{\textsc{Matlab}\xspace}
\newcommand{\Octave}{\textsc{Octave}\xspace}
\newcommand{\Python}{\textsc{Python}\xspace}

\newcommand{\prob}[1]{\bigskip\noindent\textbf{#1}\quad }

\newcommand{\chapexers}[2]{\prob{Chapter #1, pages #2, Exercises:}}
\newcommand{\exer}[2]{\prob{Exercise #1}}

\newcommand{\pts}[1]{(\emph{#1 pts}) }
\newcommand{\epart}[1]{\medskip\noindent\textbf{#1)}\quad }
\newcommand{\ppart}[1]{\,\textbf{#1)}\quad }

\newcommand*\circled[1]{\tikz[baseline=(char.base)]{
            \node[shape=ellipse,draw,inner sep=2pt] (char) {#1};}}


\begin{document}
\scriptsize \noindent Math 615 NADE (Bueler) \hfill 28 March 2023
\normalsize

\medskip\bigskip

\Large\centerline{\textbf{Assignment \#7}}
\large
\bigskip

\centerline{\textbf{Due Monday, 10 April 2023, at the start of class}}
\bigskip
\normalsize

\thispagestyle{empty}

\bigskip
Please read textbook\footnote{R.~J.~LeVeque, \emph{Finite Difference Methods for Ordinary and Partial Diff.~Eqns.}, SIAM Press 2007} Chapters 6 and 7.  Within this material we are de-emphasizing the discussion of multistep methods, so full understanding of sections 5.9, 6.4, 7.3, and 7.6.1 is not expected.  However, \emph{actually reading} these Chapters is going to be important to success on this and later Assignments.


\medskip
\prob{Problem P30.}  Consider the ``$\theta$-methods'' for $u' = f(t,u)$, namely
   $$U^{n+1} = U^n + k\Big[(1-\theta)f(t_n,U^n) + \theta f(t_{n+1},U^{n+1})\Big],$$
where $0\le \theta \le 1$ is a fixed parameter.  I did all parts of this problem by hand.

\epart{a} Cases $\theta = 0,1/2,1$ are all familiar methods.  Name them.

\epart{b} Find the (absolute) stability regions for $\theta = 0,1/4,1/2,3/4,1$.  (\emph{Hint.}  Write the complex number $z=k\lambda$ as $z=x+iy$.  Find the circles!)

\epart{c} Show that the $\theta$-methods are A-stable for $\theta \geq 1/2$.


\prob{Problem P31.}  Consider this Runge-Kutta method, a one-step and implicit interpretation of the multistep midpoint method:
\begin{align*}
U^* &= U^n + \frac{k}{2} f\big(t_n + k/2, U^*\big),\\
U^{n+1} &= U^n + k f\big(t_n + k/2, U^*\big).
\end{align*}
The first stage is backward Euler to determine an approximation to the value at the midpoint in time.  The second stage is a midpoint method using this value.

\epart{a}  Determine the order of accuracy of this method.  That is, compute the truncation error accurately enough to know the power $p$ in $\tau=O(k^p)$.

\epart{b}  Determine the stability region.  Is this method A-stable?  Is it L-stable?


\prob{Problem P32.}  Subsection 5.9.4 explains why the explicit trapezoid rule (5.53) is sometimes called a \emph{predictor-corrector} method.  A full Euler step is used to estimate (``predict'') the new solution value, and then a formerly-implicit method, the original trapezoid method, is used to actually take the step, which ``corrects'' the prediction.  This problem attempts to construct a predictor-corrector method with higher order.

\epart{a}  On page 132 (section 5.9) there is a list of explicit Adams-Bashforth and implicit Adams-Moulton multistep methods.  Suppose we cut a step into two parts, with stage steps $k/2$.  From $U^n$ suppose we take a  forward Euler step of length $k/2$ to give $U^*$ at $t_n+k/2$.  Then we use $U^n$ and $U^*$ in the second-order (``2-step'') Adams-Bashforth scheme to give $U^\dagger$ at $t_n+k=t_{n+1}$.  The value $U^\dagger$ is the predicted value at $t_{n+1}$.  Then we use the second-order (``2-step'') Adams-Moulton scheme, using known values $U^n,U^*,U^\dagger$ on the right-hand side, as the corrector to give $U^{n+1}$.  Write down these formulas.  (You don't need to implement this scheme, but your formulas should make it clear how to do so.)

\epart{b}  The scheme you constructed in part \textbf{(a)} is a one-step, multi-stage Runge-Kutta scheme.  Write down its Butcher tableau.  (\emph{Hint.}  Be careful!  You really can write the scheme in form (5.34), where $Y_1=U^n$, $Y_2=U^*$, $Y_3=U^\dagger$.)

\epart{c}  Following what is described at the beginning of section 7.6.2, apply the scheme in part \textbf{(a)} to the test equation $u'=\lambda u$ and write the scheme as $U^{n+1} = R(z) U^n$ for $z=\lambda k$.  What is $R(z)$?  From the form of $R(z)$, what is the order of the truncation error?  (That is, what is $p$ in $\tau=O(k^p)$?)  Would you recommend this method?

\epart{d} Generate a plot of the stability region of the method; see the advice in section 7.6.2 on how to do this.


\prob{Problem P33.}  For a famously stiff problem, consider the heat PDE
\begin{equation}
u_t = u_{xx}  \label{heat}
\end{equation}
Here $u(t,x)$ is the temperature in a rod of length one ($0 \le x \le 1$) and we set boundary temperatures to zero ($u(t,0)=0$ and $u(t,1)=0$).  For an initial temperature distribution we set one part hotter than the rest:
    $$u(0,x) = \begin{cases} 1, & 0.25 < x < 0.5, \\
                             0, & \text{otherwise}.\end{cases}$$
Suppose we seek $u(1,x)$, i.e.~we set $t_f = 1$.

We apply the \emph{method of lines} (MOL) to \eqref{heat}.  That is, we discretize the spatial ($x$) derivatives using the notation from Chapter 2.  Specifically, use $m+1$ subintervals, let $h=1/(m+1)$, and let $x_j = j h$ for $j=0,1,2,\dots,m+1$.  Now $U_j(t) \approx u(t,x_j)$.  By eliminating unknowns $U_0=0$ and $U_{m+1}=0$, and keeping the time derivatives as ordinary derivatives, from \eqref{heat} we get a linear ODE system of dimension $m$,
\begin{equation}
U(t)' = A U(t)  \label{mol}
\end{equation}
where $U(t) \in \RR^m$ and $A$ is \emph{exactly} the matrix in the textbook's equation (2.10).  For a given $m$, note $U(0)$ is computed from the above formula for $u(0,x)$.

\epart{a}  Implement both forward and backward Euler on \eqref{mol}.  For BE, store $A$ using \texttt{sparse} storage and solve the equation using backslash or another linear solver which will automatically detect that the matrix is tridiagonal and solve it efficiently.

\epart{b}  Now consider the $m=100$ case.  For BE, compute and show the solution using $N=100$ time steps.  For FE, $N=100$ will generate extraordinary explosion.  (Confirm this but don't show it.)  Determine the largest-possible absolutely-stable time step $k$ from the eigenvalues of $A$ and the stability region of FE.  Finally, compare the computational costs of the two runs by counting floating-point multiplications.\footnote{For an $m\times m$ tridiagonal matrix $A$, $A v$ costs $3m$ multiplications while $A\backslash v$ costs $5m$ multiplications.}  You will conclude that an implicit is indeed effective in this case.
\end{document}
