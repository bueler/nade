\documentclass[12pt]{amsart}
%prepared in AMSLaTeX, under LaTeX2e
\addtolength{\oddsidemargin}{-.6in} 
\addtolength{\evensidemargin}{-.6in}
\addtolength{\topmargin}{-.4in}
\addtolength{\textwidth}{1.2in}
\addtolength{\textheight}{.6in}

\renewcommand{\baselinestretch}{1.05}

\usepackage{verbatim,fancyvrb}
\usepackage{soul}
\usepackage{palatino}

\newtheorem*{thm}{Theorem}
\newtheorem*{defn}{Definition}
\newtheorem*{example}{Example}
\newtheorem*{problem}{Problem}
\newtheorem*{remark}{Remark}

\newcommand{\mtt}{\texttt}
\usepackage{alltt,xspace}
\newcommand{\mfile}[1]
{\medskip\begin{quote}\scriptsize \begin{alltt}\input{#1.m}\end{alltt} \normalsize\end{quote}\medskip}

\usepackage[final]{graphicx}
\newcommand{\mfigure}[1]{\includegraphics[height=2.5in,
width=3.5in]{#1.eps}}
\newcommand{\regfigure}[2]{\includegraphics[height=#2in,
keepaspectratio=true]{#1.eps}}
\newcommand{\widefigure}[3]{\includegraphics[height=#2in,
width=#3in]{#1.eps}}

\usepackage{amssymb}

\usepackage[pdftex, colorlinks=true, plainpages=false, linkcolor=black, citecolor=red, urlcolor=red]{hyperref}

% macros
\newcommand{\br}{\mathbf{r}}
\newcommand{\bv}{\mathbf{v}}
\newcommand{\bx}{\mathbf{x}}
\newcommand{\by}{\mathbf{y}}

\newcommand{\CC}{\mathbb{C}}
\newcommand{\RR}{\mathbb{R}}
\newcommand{\ZZ}{\mathbb{Z}}

\newcommand{\eps}{\epsilon}
\newcommand{\grad}{\nabla}
\newcommand{\lam}{\lambda}
\newcommand{\lap}{\triangle}

\newcommand{\ip}[2]{\ensuremath{\left<#1,#2\right>}}

%\renewcommand{\det}{\operatorname{det}}
\newcommand{\onull}{\operatorname{null}}
\newcommand{\rank}{\operatorname{rank}}
\newcommand{\range}{\operatorname{range}}

\newcommand{\Julia}{\textsc{Julia}\xspace}
\newcommand{\Matlab}{\textsc{Matlab}\xspace}
\newcommand{\Octave}{\textsc{Octave}\xspace}
\newcommand{\Python}{\textsc{Python}\xspace}

\newcommand{\prob}[1]{\bigskip\noindent\textbf{#1}\quad }

\newcommand{\chapexers}[2]{\prob{Chapter #1, pages #2, Exercises:}}
\newcommand{\exer}[2]{\prob{Exercise #1}}

\newcommand{\pts}[1]{(\emph{#1 pts}) }
\newcommand{\epart}[1]{\medskip\noindent\textbf{#1)}\quad }
\newcommand{\ppart}[1]{\,\textbf{#1)}\quad }

\newcommand*\circled[1]{\tikz[baseline=(char.base)]{
            \node[shape=ellipse,draw,inner sep=2pt] (char) {#1};}}


\begin{document}
\scriptsize \noindent Math 615 NADE (Bueler) \hfill 15 February, 2023
\normalsize

\medskip\bigskip

\Large\centerline{\textbf{Assignment \#4}}
\large
\bigskip

\centerline{\textbf{Due Wednesday, 1 March 2023, at the start of class} (\emph{revised})}
\bigskip
\normalsize

\thispagestyle{empty}

\bigskip
Please read sections 2.12--2.21 and 3.1--3.4 from the textbook.\footnote{R.~J.~LeVeque, \emph{Finite Difference Methods for Ordinary and Partial Diff.~Eqns.}, SIAM Press 2007}  Two of the problems below require you to remember that you can find the general solutions of constant-coefficient, linear, homogeneous ODEs \emph{by hand} when needed.

\medskip
\prob{Problem P18.}  (\emph{A sometimes ill-posed boundary value problem.}) 

\epart{a}  Consider the following linear BVP with Dirichlet boundary conditions:
\begin{equation}
u''(x) +  u(x) = 0 \quad \text{for $a< x< b$}, \qquad u(a)=\alpha, \qquad u(b)=\beta.  \label{p12odebvp}
\end{equation}
(\emph{This equation arises from linearizing the pendulum equation (2.75), for example.})  Write a \Matlab/etc.~finite difference code to solve this problem.

\epart{b}  Determine the exact solution to the problem when $a = 0, b = 1, \alpha = 2, \beta = 3$.  Test your code from part \textbf{a)} using this solution, including a demonstration of convergence at the optimal rate (\emph{which is what?}) as $h \to 0$.  Generate a convergence figure.

\epart{c}  Let $a=0$ and $b=\pi$.  For what values of $\alpha$ and $\beta$ does BVP \eqref{p12odebvp} have solutions?  Sketch a family of solutions in a case where there are infinitely-many solutions.


\prob{Problem P19.}  (\emph{Nonlinear pendulum.})   Write a program to solve the BVP for the nonlinear pendulum as discussed in the text, i.e.~problem (2.77), using the Newton iteration strategy outlined in subsection 2.16.1.  Reproduce Figures 2.4(b) and 2.5, for which $T=2\pi,\alpha=\beta=0.7$.


\prob{Problem P20.}  Recall the ODE BVPs from \textbf{P11} on Assignment \#2:
\begin{equation*}
u''(x) + p(x)\, u'(x) + q(x)\, u(x) = f(x), \quad u(x_L) = \alpha, \quad u(x_R) = \beta.
\end{equation*}

\epart{a}  Consider the case $x_L=0,x_R=1,\alpha=1,\beta=0,p=-20,q=0$, and $f(x)=0$.  Confirm that an exact solution to this problem is
    $$u(x) = 1 - \frac{1 - e^{20x}}{1-e^{20}}.$$
Is it the only solution?

\epart{b}  Solve the problem in part \textbf{a)} numerically using centered finite differences, $h=1/(m+1)$ equal spacing, and $m=3,5,10,20,50,200,1000$ interior points.  Put all these numerical solutions, and the exact solution from \textbf{a)}, on one figure, with decent labeling.  (Use \texttt{legend} or similar.)

\epart{c}  Observe that the solutions for small values of $m$ are poor but the high $m$ solutions all basically agree.  Why do you think that small $m$ values are problematic here, though they are not for the problem solved in section 2.4?   Write a few sentences, perhaps based on a bit of research into section 2.17 or wikipedia pages. (\emph{Hint.}  Observe that the exact solution in part \textbf{a)} mostly does not ``feel'' one of the boundary conditions.  Also consider the $p=+20$ case, with other data the same.  What is the exact solution to the reduced equation $pu'=0$ and what boundary conditions does it need?)


\prob{Problem P21.}  (\emph{This problem is updated.})  \quad (\emph{Poisson equation on the unit square.})

\epart{a}  Based on the ideas in sections 3.1--3.3, namely centered finite differences, write a \Matlab/etc.~code that solves Poisson equation (3.5) on the unit square $0\le x \le 1, 0 \le y \le 1$, with zero Dirichlet boundary conditions.  Allow the user to set the function $f=f(x,y)$.  You may, as usual, use \Matlab's backslash, or equivalent, to solve the linear system.

\medskip
\noindent \emph{Hint.}  Start from my code \texttt{heat2d.m} at

\url{https://bueler.github.io/nade/assets/codes/heat2d.m}.

\epart{b}  Find an appropriate nonzero exact solution that allows you to verify that the code converges at the theoretically-expected rate.  That is, show that $\|E^h\|_2 \to 0$, as $h\to 0$, at the rate $O(h^2)$ if $\Delta x = \Delta y = h = 1 / (m+1)$.

\epart{c}  Use \Matlab's \texttt{spy}, or similar, to show the sparsity pattern of the matrix $A^h$ for $m=5$.  Also confirm in that case that the matrix has the form shown by equation (3.12).


\end{document}
