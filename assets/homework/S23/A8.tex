\documentclass[12pt]{amsart}
%prepared in AMSLaTeX, under LaTeX2e
\addtolength{\oddsidemargin}{-.6in} 
\addtolength{\evensidemargin}{-.6in}
\addtolength{\topmargin}{-.5in}
\addtolength{\textwidth}{1.2in}
\addtolength{\textheight}{1.0in}

\renewcommand{\baselinestretch}{1.05}

\usepackage{verbatim,fancyvrb}
\usepackage{soul}
\usepackage{palatino}

\newtheorem*{thm}{Theorem}
\newtheorem*{defn}{Definition}
\newtheorem*{example}{Example}
\newtheorem*{problem}{Problem}
\newtheorem*{remark}{Remark}

\newcommand{\mtt}{\texttt}
\usepackage{alltt,xspace}
\newcommand{\mfile}[1]
{\medskip\begin{quote}\scriptsize \begin{alltt}\input{#1.m}\end{alltt} \normalsize\end{quote}\medskip}

\usepackage[final]{graphicx}
\newcommand{\mfigure}[1]{\includegraphics[height=2.5in,
width=3.5in]{#1.eps}}
\newcommand{\regfigure}[2]{\includegraphics[height=#2in,
keepaspectratio=true]{#1.eps}}
\newcommand{\widefigure}[3]{\includegraphics[height=#2in,
width=#3in]{#1.eps}}

\usepackage{amssymb}

\usepackage[pdftex, colorlinks=true, plainpages=false, linkcolor=black, citecolor=red, urlcolor=red]{hyperref}

% macros
\newcommand{\bb}{\mathbf{b}}
\newcommand{\br}{\mathbf{r}}
\newcommand{\bu}{\mathbf{u}}
\newcommand{\bv}{\mathbf{v}}
\newcommand{\bx}{\mathbf{x}}
\newcommand{\by}{\mathbf{y}}

\newcommand{\bU}{\mathbf{U}}

\newcommand{\CC}{\mathbb{C}}
\newcommand{\RR}{\mathbb{R}}
\newcommand{\ZZ}{\mathbb{Z}}

\newcommand{\eps}{\epsilon}
\newcommand{\grad}{\nabla}
\newcommand{\lam}{\lambda}
\newcommand{\lap}{\triangle}

\newcommand{\ip}[2]{\ensuremath{\left<#1,#2\right>}}

%\renewcommand{\det}{\operatorname{det}}
\newcommand{\onull}{\operatorname{null}}
\newcommand{\rank}{\operatorname{rank}}
\newcommand{\range}{\operatorname{range}}

\newcommand{\Julia}{\textsc{Julia}\xspace}
\newcommand{\Matlab}{\textsc{Matlab}\xspace}
\newcommand{\Octave}{\textsc{Octave}\xspace}
\newcommand{\Python}{\textsc{Python}\xspace}

\newcommand{\prob}[1]{\bigskip\noindent\textbf{#1}\quad }

\newcommand{\chapexers}[2]{\prob{Chapter #1, pages #2, Exercises:}}
\newcommand{\exer}[2]{\prob{Exercise #1}}

\newcommand{\pts}[1]{(\emph{#1 pts}) }
\newcommand{\epart}[1]{\medskip\noindent\textbf{#1)}\quad }
\newcommand{\ppart}[1]{\,\textbf{#1)}\quad }

\newcommand*\circled[1]{\tikz[baseline=(char.base)]{
            \node[shape=ellipse,draw,inner sep=2pt] (char) {#1};}}


\begin{document}
\scriptsize \noindent Math 615 NADE (Bueler) \hfill 8 April 2023
\normalsize

\medskip\bigskip

\Large\centerline{\textbf{Assignment \#8}}
\large
\bigskip

\centerline{\textbf{Due Wednesday, 19 April 2023, at the start of class}}
\bigskip
\normalsize

\thispagestyle{empty}

\bigskip
Please read textbook\footnote{R.~J.~LeVeque, \emph{Finite Difference Methods for Ordinary and Partial Diff.~Eqns.}, SIAM Press 2007} Chapters 8, 9, and 10.  This assignment is mostly Chapter 9.


\medskip
\prob{Problem P34.}  Consider the heat equation $u_t = D\, u_{xx}$ for $D>0$ constant, $x\in [0,1]$, and Dirichlet boundary conditions $u(t,0)=0$ and $u(t,1)=0$.  Suppose we have initial condition $u(0,x) = \sin(5\pi x)$.

\epart{a}  Confirm that
    $$u(t,x) = e^{-25 \pi^2 D t} \sin(5 \pi x)$$
is an exact solution.  (\emph{It is \emph{the} exact solution, but you do not have to show this.})

\epart{b}  Implement the backward Euler (BE) method, as applied to MOL ODE system (9.10), to solve this heat equation problem.  Specifically, use diffusivity $D = 1/20$ and final time $t_f=0.1$.  Note that you do not need to use Newton's method to solve the implicit equation, a linear system, but you should use sparse storage and \Matlab's backslash or similar.  Feel free to reuse or modify code from \textbf{P33} on Assignment \#7.

\epart{c}  Suppose we set $k=h$ for the ``refinement path''.  (\emph{Of course, for BE stability \emph{does not} constrain our refinement path.})  What do you expect for the convergence rate $O(h^p)$?  Then measure it by using the exact solution from \textbf{a)}, at the final time, and the infinity norm $\|\cdot\|_\infty$, and $h=0.02, 0.01,0.005,0.002,0.001,0.0005$.  Make a log-log convergence plot of $h$ versus the error.

\epart{d}  Repeat parts \textbf{b)} and \textbf{c)} but with the trapezoidal rule, i.e.~Crank-Nicolson (CN) equation (9.6).  Use the same refinement path.  Add the result to the same plot; turn in codes for BE and CN, but only one well-designed log-log convergence plot.


\prob{Problem P35.}   Consider the following scheme, which applies centered differences to both sides of the heat equation $u_t=u_{xx}$:
    $$U_j^{n+2} = U_j^n + \frac{2k}{h^2}(U_{j-1}^{n+1} - 2U_j^{n+1} +
U_{j+1}^{n+1}).$$
This is called the \emph{Richardson} method.  (\emph{L.~F.~Richardson did many things more important than, and successful than, inventing this scheme!})

\epart{a} Compute the truncation error to determine the order of accuracy of this method, in space and time.  The answer will be in form $\tau(t,x) = O(k^p + h^q)$; determine $p,q$.

\epart{b} Derive the method by applying the midpoint ODE method, equation (5.23), to the MOL ODE system (9.10).   Also, find the region of absolute stability of the midpoint method (5.23); it is in the textbook.  Is the method likely to generate reasonable results?  Why or why not?

\epart{c} Implement the method on the same problem as in \textbf{P34}.  Was this successful?

\epart{d} Do a von Neumann stability analysis of this scheme.  What do you conclude?


\prob{Problem P36.}  Consider the Jacobi iteration\footnote{The Jacobi iteration, equation (4.4) in the textbook, was covered in the ``Classical iterative methods'' slides \quad \href{https://bueler.github.io/nade/assets/slides/iterative.pdf}{\texttt{bueler.github.io/nade/assets/slides/iterative.pdf}}} for the linear system $A\bu=\bb$ arising from a centered FD approximation of the boundary value problem $u''(x) = f(x)$.  Here $A$ is the matrix in equation (2.10); there is \emph{no} need to rederive it.  Show that this iteration can be interpreted as forward Euler time-stepping applied to a heat equation MOL system like (9.10), but \emph{with time step} $k = \frac{1}{2} h^2$.  Specifically, the MOL equations are those arising from a centered \emph{spatial} FD discretization of $u_t(t,x) = u_{xx}(t,x) - f(x)$.

\medskip \noindent
\emph{Comment 1}.  No implementations or coding is needed for this problem.

\medskip \noindent
\emph{Comment 2}.  The solution of the time-dependent heat equation decays to the steady state solution, that is, to the solution of $u'' = f$.  (This assumes steady boundary values.)  However, while marching to steady state with an explicit method is one way to solve the steady-state boundary value problem, it is a very inefficient way.  (Not recommended!)  Instead, just focus on solving the system $A\bu=\bb$ quickly.

\end{document}
