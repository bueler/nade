\documentclass[11pt]{amsart}
%prepared in AMSLaTeX, under LaTeX2e
\addtolength{\oddsidemargin}{-.65in}
\addtolength{\evensidemargin}{-.65in}
\addtolength{\topmargin}{-.6in}
\addtolength{\textwidth}{1.4in}
\addtolength{\textheight}{1.3in}
\newcommand{\normalspacing}{\renewcommand{\baselinestretch}{1.05}
        \tiny\normalsize}

\newtheorem*{thm}{Theorem}
\newtheorem*{lem}{Lemma}
\newtheorem*{defn}{Definition}
\newtheorem*{example}{Example}
\newtheorem*{problem}{Problem}
\newtheorem*{remark}{Remark}

\usepackage{amssymb,fancyvrb,alltt,xspace}

\VerbatimFootnotes

\newcommand{\mtt}{\texttt}

\usepackage{fancyvrb}
\newcommand{\mfile}[1]{
\begin{quote}
\bigskip
%\VerbatimInput[frame=single]{#1}
\VerbatimInput[frame=single,label=\fbox{\normalsize \textsl{\,#1\,}},fontfamily=courier,fontsize=\scriptsize]{#1}
\end{quote}
}

\usepackage[final]{graphicx}
\newcommand{\mfigure}[1]{\includegraphics[width=3.2in,
keepaspectratio=true]{#1.eps}}
\newcommand{\widefigure}[2]{\includegraphics[width=#2in,
keepaspectratio=true]{#1.eps}}


% macros
\newcommand{\CC}{\mathbb{C}}
\newcommand{\Div}{\nabla\cdot}
\newcommand{\eps}{\epsilon}
\newcommand{\grad}{\nabla}
\newcommand{\ZZ}{\mathbb{Z}}
\newcommand{\ip}[2]{\ensuremath{\left<#1,#2\right>}}
\newcommand{\lam}{\lambda}
\newcommand{\lap}{\triangle}
\newcommand{\RR}{\mathbb{R}}

\newcommand{\tb}{\textsc{Morton \& Mayers 2nd ed}}
\newcommand{\pexer}[2]{\bigskip\noindent\textbf{#1.} (Exercise #2 in \tb.)}
\newcommand{\prob}[1]{\bigskip\noindent\large\textbf{#1.}\normalsize }
\newcommand{\apart}[1]{\quad \textbf{(#1)} \quad }
\newcommand{\ppart}[1]{\medskip\noindent\textbf{(#1)} \quad }
\newcommand{\note}[1]{[\scriptsize #1 \normalsize]}

\newcommand{\Matlab}{\textsc{Matlab}\xspace}
\newcommand{\Octave}{\textsc{Octave}\xspace}
\newcommand{\pylab}{\textsc{pylab}\xspace}
\newcommand{\MOP}{\textsc{MOP}\xspace}


\begin{document}
\scriptsize \phantom{bob} \vspace{-0.3in}
\noindent Math 615 Numerical Analysis of DEs; \today \hfill \Large Name:\underline{\phantom{DLFJD SFLKJSD sdfa}}

\thispagestyle{empty}
\bigskip

\Large\centerline{\textbf{Project Evaluation}}
\normalsize\normalspacing

\bigskip

\begin{quote}
As announced, Version 1.0 is 5\% and Version 2.0 is 10\% of your course grade.
\end{quote}

\vspace{0.3in}

\noindent\rule{\textwidth}{0.7mm}
\bigskip

\large\centerline{\textbf{VERSION 1.0}}
\bigskip

\noindent \underline{5 points: format (\emph{5 to 10 pages, required section headings, 3 references})}
\vfill

\noindent \underline{5 points: coverage (\emph{drafted the major parts: intro of problem, num.~analysis, computation})}
\vfill

\noindent \underline{5 points: correctness/reasonableness of plan}
\vfill

\noindent \underline{5 points: quality of material so far}
\vfill

\hfill TOTAL = \phantom{foo bar} / 20

\newpage


\large\centerline{\textbf{SUGGESTIONS FOR V2.0}}
\bigskip

\vspace{3.0in}

\noindent\rule{\textwidth}{0.7mm}
\bigskip

\large\centerline{\textbf{VERSION 2.0}}
\bigskip

\noindent \underline{15 points: correctness and quality of numerical analysis}
\vfill

\noindent \underline{15 points: correctness and quality of practical computation}
\vfill

\noindent \underline{10 points: everything else (\emph{esp.~exposition of context, format})}
\vfill

\hfill TOTAL = \phantom{foo bar} / 40

\end{document}

