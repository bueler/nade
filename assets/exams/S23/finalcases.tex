\documentclass[12pt]{amsart}
\addtolength{\topmargin}{-0.6in} % usually -0.25in
\addtolength{\textheight}{1.1in} % usually 1.25in
\addtolength{\oddsidemargin}{-0.7in}
\addtolength{\evensidemargin}{-0.7in}
\addtolength{\textwidth}{1.5in} %\setlength{\parindent}{0pt}

\newcommand{\normalspacing}{\renewcommand{\baselinestretch}{1.05}\tiny\normalsize}
\newcommand{\bigspacing}{\renewcommand{\baselinestretch}{1.13}\tiny\normalsize}
\newcommand{\tablespacing}{\renewcommand{\baselinestretch}{1.0}\tiny\normalsize}
\normalspacing

% macros
\usepackage{amssymb,xspace}
\usepackage{tikz}
\usepackage[pdftex,colorlinks=true]{hyperref}


\newtheorem*{thm}{Theorem}
\newtheorem*{lem}{Lemma}

\newcommand{\mtt}{\texttt}
\newcommand{\mtl}[1]{{\texttt{>>#1}}}
\usepackage{alltt}
\usepackage{fancyvrb}

\newcommand{\bu}{\mathbf{u}}
\newcommand{\bv}{\mathbf{v}}

\newcommand{\CC}{{\mathbb{C}}}
\newcommand{\RR}{{\mathbb{R}}}
\newcommand{\ZZ}{{\mathbb{Z}}}
\newcommand{\ZZn}{{\mathbb{Z}}_n}
\newcommand{\NN}{{\mathbb{N}}}

\newcommand{\eps}{\epsilon}
\newcommand{\grad}{\nabla}
\newcommand{\lam}{\lambda}
\newcommand{\ip}[2]{\mathrm{\left<#1,#2\right>}}
\newcommand{\erf}{\operatorname{erf}}

\renewcommand{\Re}{\operatorname{Re}}
\renewcommand{\Im}{\operatorname{Im}}
\newcommand{\Arg}{\operatorname{Arg}}

\newcommand{\Span}{\operatorname{span}}
\newcommand{\rank}{\operatorname{rank}}
\newcommand{\range}{\operatorname{range}}
\newcommand{\trace}{\operatorname{tr}}
\newcommand{\Null}{\operatorname{null}}

\newcommand{\Matlab}{\textsc{Matlab}\xspace}
\newcommand{\Octave}{\textsc{Octave}\xspace}
\newcommand{\pylab}{\textsc{pylab}\xspace}
\newcommand{\longMOP}{\textsc{Matlab}\big|\textsc{Octave}\big|\textsc{pylab}\xspace}
\newcommand{\MOP}{\textsc{M}\big|\textsc{O}\big|\textsc{p}\xspace}

\newcommand{\prob}[1]{\bigskip\noindent\large\textbf{#1.} \normalsize}
\newcommand{\bookprob}[1]{\bigskip\noindent\large\textbf{Exercise #1.} \normalsize}
\newcommand{\probpart}[1]{\smallskip\noindent\textbf{(#1)}\quad }
\newcommand{\aprobpart}[1]{\textbf{(#1)}\quad }


\newcommand*\circled[1]{\tikz[baseline=(char.base)]{
            \node[shape=circle,draw,inner sep=2pt] (char) {#1};}}


\begin{document}
\scriptsize \noindent Math 615 NADE (Bueler) \hfill Thursday, 4 May 2023
\thispagestyle{empty}

\bigskip
\Large\textbf{\centerline{Final Exam: 2 Short Essays on Algorithms}}

\bigskip
\large\textbf{\centerline{Thursday, 4 May 2023, 10:15am--12:15pm, Brooks 302}}

\normalsize
\bigskip\bigskip
The in-class Final Exam will be a bit different, but short.  I would like you to summarize the finite difference numerical analysis of two of the problems we have seen during the semester, that is, I would like to write two \emph{Case Summaries}.  We have covered them in lecture and homework, and so this is an exam not a project.  Each Case Summary should have length less than a page, and it should follow the structure I describe below.  In particular, please

\medskip
\centerline{\textbf{\underline{choose 2 cases} from the 6 listed at the bottom.}}

\medskip
\noindent For each one, write a Case Summary on a separate sheet for each of your two topics.

\smallskip
\textbf{You may not bring notes to the final}, but you are \textbf{strongly encouraged to practice} your planned Case Summaries.  Make sure to read the indicated sections of the textbook for your topics.  Feel free to get feedback on your drafts from other students or faculty/friends/family/pets.  Then think through, in preparation, how you will remember enough detail so as to recreate the Summary during the Final Exam itself.

\smallskip
Each summary should be less than a page, though perhaps close to a page if you write large, and equations will be an important part of the summary.  Please a few (e.g.~at most 4) sketches as needed to communicate the problem and the numerical method.  However, please write complete sentences for the major ideas; a Case Summary which is only equations or sketches cannot achieve a good grade.

\smallskip
Your goal is to write something like a neat white-board summary of how you would solve the problem numerically.  Don't assume the reader is an expert on the topic; the assumed audience would be something like another student in the class who had not yet seen that particular problem before.

\vspace{0.2in}
\noindent {\large \textbf{The 6 Cases.}}

\newcommand{\ecomment}[1]{ \hfill \mbox{\emph{(#1)}} }

\smallskip
\renewcommand{\labelenumi}{\circled{\arabic{enumi}}}
\begin{enumerate}
\setlength{\itemsep}{4pt}
\item FIXME
\item FIXME Quasi-Newton methods, with symmetric rank-one as the example. \ecomment{section 12.3}
\end{enumerate}

\end{document}

