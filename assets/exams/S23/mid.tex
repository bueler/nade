\documentclass[11pt]{amsart}
%\pagestyle{empty} 
\setlength{\topmargin}{-0.5in} % usually -0.25in
\addtolength{\textheight}{1.2in} % usually 1.25in
\addtolength{\oddsidemargin}{-0.8in}
\addtolength{\evensidemargin}{-0.8in}
\addtolength{\textwidth}{1.6in} %\setlength{\parindent}{0pt}

\newcommand{\normalspacing}{\renewcommand{\baselinestretch}{1.1}\tiny\normalsize}
\normalspacing

% macros
\usepackage{amssymb,xspace,alltt,verbatim}
\usepackage[final]{graphicx}
\usepackage[pdftex,colorlinks=true]{hyperref}
\usepackage{fancyvrb}
\usepackage{tikz}

\newtheorem*{lem*}{Lemma}

\newcommand{\bs}{\mathbf{s}}
\newcommand{\bu}{\mathbf{u}}
\newcommand{\bv}{\mathbf{v}}
\newcommand{\bx}{\mathbf{x}}
\newcommand{\bbf}{\mathbf{f}}

\newcommand{\CC}{{\mathbb{C}}}
\newcommand{\RR}{{\mathbb{R}}}
\newcommand{\eps}{\epsilon}
\newcommand{\ZZ}{{\mathbb{Z}}}
\newcommand{\ZZn}{{\mathbb{Z}}_n}
\newcommand{\NN}{{\mathbb{N}}}
\newcommand{\ip}[2]{\mathrm{\left<#1,#2\right>}}

\renewcommand{\Re}{\operatorname{Re}}
\renewcommand{\Im}{\operatorname{Im}}

\newcommand{\Log}{\operatorname{Log}}

\newcommand{\grad}{\nabla}

\newcommand{\Matlab}{\textsc{Matlab}\xspace}
\newcommand{\Octave}{\textsc{Octave}\xspace}
\newcommand{\pylab}{\textsc{pylab}\xspace}

\newcommand{\prob}[1]{\bigskip\noindent\textbf{#1.} }
\newcommand{\pts}[1]{(\emph{#1 pts})}

\newcommand{\probpts}[2]{\prob{#1} \pts{#2}}
\newcommand{\ppartpts}[2]{\textbf{(#1)} \pts{#2}}
\newcommand{\epartpts}[2]{\medskip\noindent \textbf{(#1)} \pts{#2}}


\begin{document}
\hfill \Large Name:\underline{\phantom{Ed Bueler really really long long long name}}
\medskip

\scriptsize \noindent Math 615 NADE (Bueler) \hfill Friday, 24 March 2023
\medskip

\Large\centerline{\textbf{Midterm Exam}}

\smallskip
\large
\begin{center}
\textbf{In class.  No textbook, notes, or internet.  70 minutes.  100 points possible.}
\end{center}

\medskip

\thispagestyle{empty}

\normalsize
\probpts{1}{10}   Use Taylor's theorem to derive the centered finite difference (FD) approximation to $u''(x)$ for an equally-spaced grid.  (\emph{Hints}.  Denote the grid spacing by $h$.  Use Taylor twice.  Combine and cancel terms.)  To state your final result, fill in the blanks at the bottom.
\vfill

{\large
$$\hspace{2.0in} u''(x) = \frac{\boxed{\begin{matrix}  \phantom{sdkljfa sdakfl asdfklj asdf asdlfkj asdf} \\ \phantom{kjads} \end{matrix}}}{\boxed{\strut\phantom{ksdjf}}} + O\left(h^{\boxed{\phantom{jf}}}\right)$$
}
\bigskip

\newpage
\probpts{2}{10}   Consider the ODE BVP
\begin{equation}
    u''(x) + 5 u(x) = f(x), \qquad u(0)=\alpha, \quad u(2) = \beta,   \tag{$\ast$}
\end{equation}
for $f(x)$ a given continuous function and $\alpha,\beta$ any real numbers.  Propose an FD scheme, on an equally-spaced grid of $m$ subintervals, for problem ($\ast$).  (\emph{Hints}.  Describe the grid.  State the main FD equation.  State/include how the boundary conditions are handled.  Make the range of indices clear in each expression.  You do not need to prove or explain anything.)
\vspace{3.5in}


\newpage
\prob{3}  \ppartpts{a}{8}   For an ODE problem and an FD scheme, for example as on the previous page, define the \emph{local truncation error} $\tau^h$.  (\emph{Hints}.  You need only \emph{define} it, not simplify or expand it.)
\vfill

\epartpts{b}{5}   Define what it means for an FD scheme to be \emph{consistent}.
\vspace{2.0in}

\epartpts{c}{5}   Suppose an FD scheme with grid spacing $h$ is applied to a linear ODE or PDE BVP.  This generates a matrix equation
    $$A^h U^h = F^h.$$
Define what it means for the scheme to be \emph{stable}.
\vspace{2.0in}


\newpage
\probpts{4}{10}  Consider a system of $m$ nonlinear equations in $m$ variables,
\begin{align*}
f_1(x_1,\dots,x_m) &= 0, \\
                   &\vdots \\
f_m(x_1,\dots,x_m) &= 0.
\end{align*}
and suppose each $f_i$ is differentiable.  We may write this system as ``$F(x) = 0$'' where $x\in \RR^m$ and $F(x)\in \RR^m$.  State \emph{Newton's method} for the equations $F(x) = 0$.  (\emph{Hints}.  The method can be stated as a simple pseudocode or as a \Matlab function; syntax is not critical.  It should start with an initial iterate $x^{[0]}$ and generate a sequence of iterates $x^{[k]}$, and have an appropriate termination criterion based on a tolerance parameter.  \emph{Required}:  Make it clear which derivatives of $F(x)$ are used, and how, using standard notation as appropriate.  \emph{Required}:  Clearly identify any linear systems that need to be solved.)
\vfill


\newpage
\prob{5}  \ppartpts{a}{5}  Compute the eigenvalues of the matrix
    $$A = \begin{bmatrix}
    3 & 0 & 0 \\
    0 & 2 & -1 \\
    0 & -8 & 0 
    \end{bmatrix}. \hspace{5.0in}$$
\vfill

\epartpts{b}{5}   If $\lambda_i$ is an eigenvalue of $A$, describe in a couple of sentences how to find a corresponding eigenvector.  (\emph{Hint.}  Use complete sentences.  \emph{Required}:  State the equation satisfied by the eigenvector.)
\vspace{2.5in}

\epartpts{c}{5}   For the matrix $A$ in part \textbf{(a)}, will the iteration
    $$y_{k+1} = A y_k$$
converge for any nonzero vector $y_0\in \RR^3$?  Explain, using precise language as appropriate.
\vspace{2.0in}


\newpage
\probpts{6}{10}  For the ODE system
    $$u'(t) = f(t,u(t)),$$
where $u(t) \in \RR^s$, and supposing an initial condition $u(t_0)=\eta$ for $\eta \in \RR^s$, write a pseudocode for the \emph{explicit midpoint (EM) method}.  The user provides a final time $t_f$ and a fixed number $N$ of steps.  (\emph{Hint.}  State the method as a pseudocode or \Matlab function; syntax is not critical.)
\vfill

\probpts{7}{10}   For the scalar ODE IVP
    $$u'(t) = t - 5 u(t), \quad u(0) = 3,$$
compute $U^1$ from one step of the \emph{backward Euler} method, assuming $k=0.5$.
\vspace{3.0in}


\newpage
\probpts{8}{7}  For a square matrix $A\in\RR^{m\times m}$, define the \emph{matrix exponential} $e^A$.
\vfill

\probpts{9}{10}  Sketch, in the $(t,u)$ plane, and showing slopes for the locations where $f(t,u(t))$ is evaluated, a picture/cartoon of a single step of the \emph{explicit trapezoid (ET) method}:
\begin{align*}
U^* &= U^n + k f(t_n,U^n) \hspace{4.0in} \\
U^{n+1} &= U^n + \frac{k}{2} \left(f(t_n,U^n) + f(t_n+k,U^*)\right)
\end{align*}
(\emph{Hint.}  Annotate your sketch carefully.)
\vfill


\newpage
\probpts{Extra Credit}{3}  Recalling the definitions and notation from problem \textbf{3}, prove the fundamental theorem of finite difference methods, essentially the Lax equivalence theorem, as it applies to ODE BVPs:
    $$O(h^p) \text{ local truncation error} \quad + \quad \text{stability} \quad \implies \quad O(h^p) \text{ numerical error}.$$
\vfill

%\newpage
%\bigskip
\hrulefill
\begin{center}
{\footnotesize
\textsc{blank space for scratch work}
}
\end{center}
\vfill

\end{document}
