\documentclass[12pt]{amsart}
\addtolength{\topmargin}{-0.6in} % usually -0.25in
\addtolength{\textheight}{1.1in} % usually 1.25in
\addtolength{\oddsidemargin}{-0.7in}
\addtolength{\evensidemargin}{-0.7in}
\addtolength{\textwidth}{1.5in} %\setlength{\parindent}{0pt}

\newcommand{\normalspacing}{\renewcommand{\baselinestretch}{1.1}\tiny\normalsize}
\newcommand{\bigspacing}{\renewcommand{\baselinestretch}{1.15}\tiny\normalsize}
\newcommand{\tablespacing}{\renewcommand{\baselinestretch}{1.0}\tiny\normalsize}
\normalspacing

% macros
\usepackage{amssymb,xspace}
\usepackage[pdftex,colorlinks=true]{hyperref}

\usepackage[final]{graphicx}
\newcommand{\regfigure}[3]{\includegraphics[height=#2in,width=#3in]{#1.eps}}

\newtheorem*{thm}{Theorem}
\newtheorem*{lem}{Lemma}

\newcommand{\mtt}{\texttt}
\newcommand{\mtl}[1]{{\texttt{>>#1}}}
\usepackage{alltt}
\usepackage{fancyvrb}

\newcommand{\bu}{\mathbf{u}}
\newcommand{\bv}{\mathbf{v}}

\newcommand{\CC}{{\mathbb{C}}}
\newcommand{\RR}{{\mathbb{R}}}
\newcommand{\ZZ}{{\mathbb{Z}}}
\newcommand{\ZZn}{{\mathbb{Z}}_n}
\newcommand{\NN}{{\mathbb{N}}}

\newcommand{\eps}{\epsilon}
\newcommand{\grad}{\nabla}
\newcommand{\lam}{\lambda}
\newcommand{\ip}[2]{\mathrm{\left<#1,#2\right>}}
\newcommand{\erf}{\operatorname{erf}}

\renewcommand{\Re}{\operatorname{Re}}
\renewcommand{\Im}{\operatorname{Im}}
\newcommand{\Arg}{\operatorname{Arg}}

\newcommand{\Span}{\operatorname{span}}
\newcommand{\rank}{\operatorname{rank}}
\newcommand{\range}{\operatorname{range}}
\newcommand{\trace}{\operatorname{tr}}
\newcommand{\Null}{\operatorname{null}}

\newcommand{\Matlab}{\textsc{Matlab}\xspace}
\newcommand{\Octave}{\textsc{Octave}\xspace}
\newcommand{\pylab}{\textsc{pylab}\xspace}
\newcommand{\longMOP}{\textsc{Matlab}\big|\textsc{Octave}\big|\textsc{pylab}\xspace}
\newcommand{\MOP}{\textsc{M}\big|\textsc{O}\big|\textsc{p}\xspace}

\newcommand{\prob}[1]{\bigskip\noindent\large\textbf{#1.} \normalsize}
\newcommand{\bookprob}[1]{\bigskip\noindent\large\textbf{Exercise #1.} \normalsize}
\newcommand{\probpart}[1]{\smallskip\noindent\textbf{(#1)}\quad }
\newcommand{\aprobpart}[1]{\textbf{(#1)}\quad }


\begin{document}
\scriptsize \noindent Math 615 NADE (Bueler) \hfill Spring 2023
\thispagestyle{empty}

\bigskip
\Large\textbf{\centerline{Review Guide for In-Class Midterm Exam}}

\Large\textbf{\centerline{on Friday, 24 March 2023}}

\bigskip
\normalsize
\normalspacing

The Midterm Exam on 22 March will cover Chapters 1, 2, 3, and 5 of the textbook,\footnote{R.~LeVeque, \emph{Finite Difference Methods \dots}, SIAM Press 2007} and also sections 4.1 and 4.2 and some content from Appendices (see below).  The Exam is \emph{closed book} and \emph{closed notes}.

On this Review Guide I state \emph{specific} material that will be covered, and material significantly different from this will \emph{not} be covered.

My goal for the Midterm Exam is to only include topics that have appeared on homework and in lecture.  The problems will be of these types: state definitions, state theorems and formulas, explain/justify theorems and formulas, give examples with certain properties, or describe or illustrate/sketch concepts.  Note that I will \emph{not} ask you to ``state definition 2.1'' or anything like that which requires remembering locations in the book.

\emph{Strongly recommended}:  Get together with other students and work through this Review Guide.  Be honest with yourself about what you do and don't know.  Talk it through and learn!  Also, please ask questions about Exam content during lectures on Monday 20 March and Wednesday 22 March.

\bigskip

\bigspacing
\noindent \textbf{Definitions and Notation}.  Be able to state and use the definition, and/or use the notation/language correctly:
\begin{itemize}
\item acronyms: ODE, PDE, IVP, BVP
\item order of a differential equation ($=$ maximum number of derivatives in it)
\item one-sided and centered finite difference approximations (pages 3--4)
\item absolute and relative error (Appendix A.1)
\item $O(h^p)$ and other ``big-oh'' notation (Appendix A.2)
\item vector norm (Appendix A.3)
\item errors in grid functions (Appendix A.4)
\item local truncation error (sections 2.5, 3.4, and 5.4: $\tau^h$ is the result of applying the scheme to the exact solution)
\item global or numerical error (section 2.6: $E_j = U_j - u(x_j)$ or $E = U - \hat U$)
\item stable method (definition 2.1 in section 2.7)
\item consistent method (section 2.8)
\item convergent method (section 2.9)
\item Poisson equation, Laplace equation, Laplacian operator (page 60, section 3.1)
\item 5-point stencil for Laplacian in 2D (section 3.2)
\item eigenvalues and eigenvectors (see Assignment \#5: $A v = \lambda v$ where $v\ne 0$)
\item spectral radius (Appendix C)
\item diagonalizable matrix (Appendix C.2)
\item matrix exponential (Appendix D.3)
\item strictly diagonally-dominant matrix (Assignment \#3 and ``Classical iterative \dots'' slides)
\item Richardson iteration for linear system $Ax=b$ (same)
\item Jacobi and Gauss-Seidel iterations (same, \emph{and} sections 4.1, 4.2)
\item Newton iteration (same, \emph{and} section 2.16)
\item $f(u,t)$ is Lipschitz continuous in $u$ (formula (5.15) in section 5.2)
\item forward Euler, backward Euler, and trapezoid methods (section 5.3)
\end{itemize}

\bigskip

\noindent \textbf{Formulas, Theorems, and Lemmas}.  Understand and remember.  Be able to illustrate theorems/lemmas with an example, or give a sketch.  Use the formula as appropriate to the situation.
\begin{itemize}
\item Taylor series (page 5)
\item fundamental theorem of finite difference methods (statement (2.22), section 2.9)
\item convergence lemma for iterations $y_{k+1} = M y_k + c$ (Assignment \#3 and ``Classical iterative \dots'' slides)
\end{itemize}

\bigskip

\noindent \textbf{Techniques}.  Understand and remember.  Be able to illustrate with an example.  Use the technique as appropriate to the situation.
\begin{itemize}
\item derive a finite difference approximation by the method of undetermined coefficients (page 7)
\item set up a finite difference scheme for a first, second, third, or fourth-order ODE BVP (sections 2.4, 2.14, 2.15, 2.16.1)
\item Newton's method (section 2.16.1)
\item set up a finite difference scheme for Poisson equation in 2D (sections 3.2, 3.3)
\item implement Neumann boundary conditions for ODE BVPs (section 2.12)
\item convert a higher-order scalar ODE into a system of first-order ODEs (example given in section 5.1)
\item take a step of Forward Euler, Backward Euler, Trapezoid, or Explicit Midpoint methods (sections 5.3 and 5.6; Explicit Midpoint = (5.30))
\end{itemize}

\medskip
\noindent \emph{Make sure you can do these techniques!}  Practice a few examples.  During the Exam the emphasis will, be on quickly setting-up or explaining techniques on paper, and not, of course, doing many steps.
\vfill


\end{document}

