\documentclass[11pt]{amsart}
%prepared in AMSLaTeX, under LaTeX2e
\addtolength{\oddsidemargin}{-.9in} 
\addtolength{\evensidemargin}{-.9in}
\addtolength{\topmargin}{-.9in}
\addtolength{\textwidth}{1.7in}
\addtolength{\textheight}{1.5in}

\renewcommand{\baselinestretch}{1.05}

\usepackage{verbatim} % for "comment" environment

\usepackage{palatino}

\usepackage[final]{graphicx}

\usepackage{tikz}
\usetikzlibrary{positioning}

\usepackage{enumitem,xspace,fancyvrb}

\newtheorem*{thm}{Theorem}
\newtheorem*{defn}{Definition}
\newtheorem*{example}{Example}
\newtheorem*{problem}{Problem}
\newtheorem*{remark}{Remark}

\DefineVerbatimEnvironment{mVerb}{Verbatim}{numbersep=2mm,frame=lines,framerule=0.1mm,framesep=2mm,xleftmargin=4mm,fontsize=\footnotesize}

% macros
\usepackage{amssymb}
\newcommand{\bA}{\mathbf{A}}
\newcommand{\bB}{\mathbf{B}}
\newcommand{\bE}{\mathbf{E}}
\newcommand{\bF}{\mathbf{F}}
\newcommand{\bJ}{\mathbf{J}}

\newcommand{\bb}{\mathbf{b}}
\newcommand{\bi}{\mathbf{i}}
\newcommand{\bj}{\mathbf{j}}
\newcommand{\bk}{\mathbf{k}}
\newcommand{\br}{\mathbf{r}}
\newcommand{\bu}{\mathbf{u}}
\newcommand{\bv}{\mathbf{v}}
\newcommand{\bw}{\mathbf{w}}
\newcommand{\bx}{\mathbf{x}}

\newcommand{\ppr}[1]{\frac{\partial #1}{\partial r}}
\newcommand{\ppt}[1]{\frac{\partial #1}{\partial t}}
\newcommand{\ppx}[1]{\frac{\partial #1}{\partial x}}
\newcommand{\ppy}[1]{\frac{\partial #1}{\partial y}}
\newcommand{\ppz}[1]{\frac{\partial #1}{\partial z}}

\newcommand{\Div}{\ensuremath{\nabla\cdot}}
\newcommand{\Curl}{\ensuremath{\nabla\times}}

\newcommand{\eps}{\epsilon}
\newcommand{\grad}{\nabla}
\newcommand{\ip}[2]{\ensuremath{\left<#1,#2\right>}}
\newcommand{\lam}{\lambda}
\newcommand{\lap}{\triangle}

\newcommand{\RR}{\mathbb{R}}
\newcommand{\ZZ}{\mathbb{Z}}
\newcommand{\prob}[1]{\bigskip\noindent\textbf{#1.}\quad }

\newcommand{\Matlab}{\textsc{Matlab}\xspace}

\newcommand{\ds}{\displaystyle}

\begin{document}
\scriptsize \noindent Math 615 NADE (Bueler) \hfill \quad \fbox{\emph{Not turned in!}}
\normalsize

\bigskip

\Large\centerline{\textbf{DRAFT Worksheet:  Stable time-steps for 2nd-order ODE IVP solvers}}
\medskip
\normalsize

\thispagestyle{empty}

\bigskip\bigskip
\noindent \textbf{METHODS.}  Consider the following three ODE IVP methods, as they apply to an autonomous ODE system $u'=f(u)$.  All have $O(k^2)$ local truncation error, and all are stated in the textbook (R.~J.~LeVeque, 2007).  Note BDF2 is the 2nd-order backward differentiation formula.

\bigskip
\renewcommand{\arraystretch}{1.4}
\begin{tabular}{lll}
\emph{method}\phantom{sdfaadsfasdfasdxxx} & \emph{book reference}\phantom{xx} & \emph{formula} \\ \hline
TR: trapezoidal         & (5.22) & $U^{n+1} = U^n + \frac{k}{2} \left(f(U^n) + f(U^{n+1})\right)$ \\
RK2: explicit midpoint  & (5.30) & $U^{n+1} = U^n + k f\left(U^n + \frac{k}{2} f(U^n)\right)$ \\
BDF2                    & p.~173 & $U^{n+2} = + \frac{4}{3} U^{n+1} - \frac{1}{3} U^n + \frac{2}{3} k f\left(U^{n+2}\right)$
\end{tabular}


\bigskip
\noindent \textbf{PROBLEMS.}  Consider the following three linear ODE systems:
\renewcommand{\labelenumi}{S\arabic{enumi}.\,}
\begin{enumerate}
\item $u' = A u$ where $u(t)\in\RR^3$ and
    $$A = \begin{bmatrix}
    2 & 2 & 2 \\
    2 & 1 & 1 \\
    2 & 1 & -5
    \end{bmatrix}$$
\item $u' = A u$ where $u(t)\in\RR^2$ and
    $$A = \begin{bmatrix}
    0 & 2 \\
    -2 & 0
    \end{bmatrix}$$
\item $u' = A u$ where $u(t)\in\RR^{25}$ and $A$ is the tridiagonal matrix for the $m=25$ case (i.e.~$h=1/26$ case) of the method-of-lines-discretization of the heat equation $u_t = u_{xx}$; this $A$ models a diffusion process.
\end{enumerate}


\bigskip
\noindent \textbf{TASKS.}
\renewcommand{\labelenumi}{\alph{enumi})\,}
\begin{enumerate}
\item For each of the METHODS, write a \Matlab etc.~code, presumably using \texttt{meshgrid} and \texttt{contour} or \texttt{contourf}, to plot the absolute stability region.
\item For each of the PROBLEMS, use \Matlab etc.~to compute the eigenvalues $\lambda_p$ of $A$.
\item For each pair (METHOD, PROBLEM), either determine that there is no stability restriction on the time step, or determine the maximum absolutely-stable time step $k_{\text{stab}}$.  For the latter cases, show a figure which has the relevant $z$-values (i.e.~$z=k_{\text{stab}} \lambda_p$) plotted on top of the stability region.
\item For each of the PROBLEMS, give expert advice: which METHOD is best and why?  (\emph{Hints}.  Generally, think of various ways in which a given method does or does not preserve ``qualities'' relevant to the particular problem.  Computational cost is a consideration too; assume your computer is weak.  See section 7.5 and think about $k_{\text{stab}}$ and $k_{\text{acc}}$.  See also sections 8.3 and 8.4.)
\end{enumerate}

\vfill
\end{document}
