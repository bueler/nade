\documentclass[11pt]{amsart}
%prepared in AMSLaTeX, under LaTeX2e
\addtolength{\oddsidemargin}{-.7in} 
\addtolength{\evensidemargin}{-.7in}
\addtolength{\topmargin}{-.5in}
\addtolength{\textwidth}{1.3in}
\addtolength{\textheight}{1.0in}

\renewcommand{\baselinestretch}{1.05}

\usepackage{verbatim} % for "comment" environment

\usepackage{palatino}

\usepackage[final]{graphicx}

\usepackage{tikz}
\usetikzlibrary{positioning}

\usepackage{enumitem,xspace,fancyvrb}

\newtheorem*{thm}{Theorem}
\newtheorem*{defn}{Definition}
\newtheorem*{example}{Example}
\newtheorem*{problem}{Problem}
\newtheorem*{remark}{Remark}

\DefineVerbatimEnvironment{mVerb}{Verbatim}{numbersep=2mm,frame=lines,framerule=0.1mm,framesep=2mm,xleftmargin=4mm,fontsize=\footnotesize}

% macros
\usepackage{amssymb}
\newcommand{\bA}{\mathbf{A}}
\newcommand{\bB}{\mathbf{B}}
\newcommand{\bE}{\mathbf{E}}
\newcommand{\bF}{\mathbf{F}}
\newcommand{\bJ}{\mathbf{J}}

\newcommand{\bb}{\mathbf{b}}
\newcommand{\bi}{\mathbf{i}}
\newcommand{\bj}{\mathbf{j}}
\newcommand{\bk}{\mathbf{k}}
\newcommand{\br}{\mathbf{r}}
\newcommand{\bu}{\mathbf{u}}
\newcommand{\bv}{\mathbf{v}}
\newcommand{\bw}{\mathbf{w}}
\newcommand{\bx}{\mathbf{x}}

\newcommand{\ppr}[1]{\frac{\partial #1}{\partial r}}
\newcommand{\ppt}[1]{\frac{\partial #1}{\partial t}}
\newcommand{\ppx}[1]{\frac{\partial #1}{\partial x}}
\newcommand{\ppy}[1]{\frac{\partial #1}{\partial y}}
\newcommand{\ppz}[1]{\frac{\partial #1}{\partial z}}

\newcommand{\Div}{\ensuremath{\nabla\cdot}}
\newcommand{\Curl}{\ensuremath{\nabla\times}}

\newcommand{\eps}{\epsilon}
\newcommand{\grad}{\nabla}
\newcommand{\ip}[2]{\ensuremath{\left<#1,#2\right>}}
\newcommand{\lam}{\lambda}
\newcommand{\lap}{\triangle}

\newcommand{\RR}{\mathbb{R}}
\newcommand{\ZZ}{\mathbb{Z}}
\newcommand{\prob}[1]{\bigskip\noindent\textbf{#1.}\quad }

\newcommand{\Matlab}{\textsc{Matlab}\xspace}

\newcommand{\ds}{\displaystyle}

\begin{document}
\scriptsize \noindent Math 615 NADE (Bueler) \hfill \quad \fbox{\emph{Not turned in!}}
\normalsize

\bigskip

\Large\centerline{\textbf{Worksheet:  Stable time-steps for 2nd-order ODE schemes}}
\medskip
\normalsize

\thispagestyle{empty}

\bigskip\bigskip
\noindent \textbf{METHODS.}\footnote{R.~J.~LeVeque, \emph{Finite Difference Methods for Ordinary and Partial Diff.~Eqns.}, SIAM Press 2007}  Consider these 3 ODE IVP methods for an autonomous ODE system $u'=f(u)$:

\medskip
\renewcommand{\arraystretch}{1.4}
\begin{tabular}{l|l|l}
\emph{method}\phantom{sdfaadsfasdfasdxxx} & \emph{book reference}\phantom{xx} & \emph{formula} \\ \hline
TR: trapezoidal         & (5.22) & $U^{n+1} = U^n + \frac{k}{2} \left(f(U^n) + f(U^{n+1})\right)$ \\
EM: explicit midpoint   & (5.30) & $U^{n+1} = U^n + k f\left(U^n + \frac{k}{2} f(U^n)\right)$ \\
AB2: Adams-Bashforth    & p.~132 & $U^{n+2} = U^{n+1} + \frac{k}{2} \left(-f\left(U^{n}\right) + 3f\left(U^{n+1}\right)\right)$
\end{tabular}

\medskip
\noindent All 3 methods have $O(k^2)$ local truncation error.  Note EM is an explicit RK2 method.

\bigskip
\noindent \textbf{PROBLEMS.}  Consider the following three linear ODE systems:

\medskip
\renewcommand{\labelenumi}{S\arabic{enumi}.\,}
\begin{enumerate}
\item $u' = A u$ where $u(t)\in\RR^3$ and

\vspace{-7mm}
{\small
    $$A = \begin{bmatrix}
    2 & 2 & 2 \\
    2 & 1 & 1 \\
    2 & 1 & -5
    \end{bmatrix}$$
}
\item $u' = A u$ where $u(t)\in\RR^2$ and

\vspace{-6mm}
{\small
    $$A = \begin{bmatrix}
    0 & 2 \\
    -2 & 0
    \end{bmatrix}$$
}
\item $u' = A u$ where $u(t)\in\RR^{25}$ and $A$ is the tridiagonal matrix shown in equation (2.10) for the $m=25$ case (i.e.~$h=1/26$ case); this $A$ approximates a second derivative in space
\end{enumerate}


\bigskip
\noindent \textbf{TASKS.}
\renewcommand{\labelenumi}{\textbf{(\alph{enumi})}\,}
\begin{enumerate}
\item For each of the one-step METHODS, find the stability function $R(z)$, for $z=k\lambda$, and write a \Matlab etc.~code, using \texttt{contourf} or \texttt{imagesc} or similar as needed, to plot the region of absolute stability.
\item For the AB2 METHOD, which is a multistep scheme, a slightly different approach is needed to plot the stability region.  Following the approach of section 7.3, use the facts that $\rho(\zeta) = \zeta^2 - \zeta$, $\sigma(\zeta) = (3/2) \zeta - (1/2)$ to compute a quadratic polynomial $\pi(\zeta; z)$.  Then you can either follow the approach of 7.6.1 or check the root condition at every point in a mesh and apply \texttt{contourf} etc.\footnote{Give this task a shot here on the worksheet.  But it will not be on homework, and don't sweat it.}
\item For each PROBLEM, compute/approximate all of the eigenvalues $\lambda_p$ of $A$.
\newcommand{\kstab}{k_{\text{stab}}}
\item For each pair (METHOD, PROBLEM), determine the maximum absolutely-stable time step $\kstab$, or explain why there is no stability restriction on the time step ($\kstab=+\infty$).  For $\kstab<\infty$ cases show a figure which has the relevant $z$-values ($z=\kstab \lambda_p$) plotted on top of the stability region.
\item For each of the PROBLEMS, give expert advice: which METHOD is best and why?  For this question think of various ways in which a given method does or does not preserve ``qualities'' relevant to the particular problem.  Computational cost is a consideration too; assume your computer is slow.  (\emph{Hints. See section 7.5 and think about $\kstab$ and $k_{\text{acc}}$.  See also sections 8.3 and 8.4; A-stable and L-stable properties are relevant.})
\end{enumerate}

\vfill
\end{document}
