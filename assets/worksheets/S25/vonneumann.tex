\documentclass[11pt]{amsart}
%prepared in AMSLaTeX, under LaTeX2e
\addtolength{\oddsidemargin}{-.9in} 
\addtolength{\evensidemargin}{-.9in}
\addtolength{\topmargin}{-.9in}
\addtolength{\textwidth}{1.7in}
\addtolength{\textheight}{1.5in}

\renewcommand{\baselinestretch}{1.05}

\usepackage{verbatim} % for "comment" environment

\usepackage{palatino}

\usepackage[final]{graphicx}

\usepackage{tikz}
\usetikzlibrary{positioning}

\usepackage{enumitem,xspace,fancyvrb}

\newtheorem*{thm}{Theorem}
\newtheorem*{defn}{Definition}
\newtheorem*{example}{Example}
\newtheorem*{problem}{Problem}
\newtheorem*{remark}{Remark}

\DefineVerbatimEnvironment{mVerb}{Verbatim}{numbersep=2mm,frame=lines,framerule=0.1mm,framesep=2mm,xleftmargin=4mm,fontsize=\footnotesize}

% macros
\usepackage{amssymb}
\newcommand{\bA}{\mathbf{A}}
\newcommand{\bB}{\mathbf{B}}
\newcommand{\bE}{\mathbf{E}}
\newcommand{\bF}{\mathbf{F}}
\newcommand{\bJ}{\mathbf{J}}

\newcommand{\bb}{\mathbf{b}}
\newcommand{\bi}{\mathbf{i}}
\newcommand{\bj}{\mathbf{j}}
\newcommand{\bk}{\mathbf{k}}
\newcommand{\br}{\mathbf{r}}
\newcommand{\bu}{\mathbf{u}}
\newcommand{\bv}{\mathbf{v}}
\newcommand{\bw}{\mathbf{w}}
\newcommand{\bx}{\mathbf{x}}

\newcommand{\ppr}[1]{\frac{\partial #1}{\partial r}}
\newcommand{\ppt}[1]{\frac{\partial #1}{\partial t}}
\newcommand{\ppx}[1]{\frac{\partial #1}{\partial x}}
\newcommand{\ppy}[1]{\frac{\partial #1}{\partial y}}
\newcommand{\ppz}[1]{\frac{\partial #1}{\partial z}}

\newcommand{\Div}{\ensuremath{\nabla\cdot}}
\newcommand{\Curl}{\ensuremath{\nabla\times}}

\newcommand{\eps}{\epsilon}
\newcommand{\grad}{\nabla}
\newcommand{\ip}[2]{\ensuremath{\left<#1,#2\right>}}
\newcommand{\lam}{\lambda}
\newcommand{\lap}{\triangle}

\newcommand{\CC}{\mathbb{C}}
\newcommand{\RR}{\mathbb{R}}
\newcommand{\ZZ}{\mathbb{Z}}
\newcommand{\prob}[1]{\bigskip\noindent\textbf{#1.}\quad }

\newcommand{\Matlab}{\textsc{Matlab}\xspace}

\newcommand{\ds}{\displaystyle}

\begin{document}
\scriptsize \noindent Math 615 NADE (Bueler) \hfill \fbox{\emph{Nothing to turn in!}}
\normalsize\medskip

\Large\centerline{\textbf{Von Neumann analysis: plug and chug}}
\medskip
\normalsize

\thispagestyle{empty}
\begin{quote}
\emph{Section 9.6 of the textbook\footnote{R.~J.~LeVeque, \emph{Finite Difference Methods for Ordinary and Partial Diff.~Eqns.}, SIAM Press 2007} introduces von Neumann analysis, but without showing how people actually \emph{do} it.  This worksheet reveals the standard style.  The textbook eventually says it clearly int the first sentences of section 10.5.}
\end{quote}


\prob{Example. FTCS on heat equation}  It is easiest to explain the idea relative to an example.  Suppose we apply the FTCS scheme to the heat equation $u_t = D u_{xx}$ with constant diffusivity $D>0$:
\begin{equation}
\frac{U_j^{n+1} - U_j^n}{k} = D \frac{U_{j-1}^n - 2 U_j^n + U_{j+1}^n}{h^2}  \label{ftcs}
\end{equation}
To find what time steps $k>0$ would be stable for a given spacing $h>0$, von Neumann substituted
\begin{equation}
U_j^n = g(\xi)^n e^{ijh\xi}  \label{ansatz}
\end{equation}
into scheme \eqref{ftcs}.  Here $\xi \in \RR$ is the \emph{wave number} for the spatial wave $e^{ijh\xi}$, in which $i=\sqrt{-1} \in \CC$ (as usual).  The scalar function $g(\xi)$ is  called the \emph{amplification factor} of the scheme.  The spatial wave is complex, but it really is a wave, and for the interval $0\le x \le 1$ the grid is $x_j = jh$ and thus
    $$e^{ijh\xi} = e^{i \xi x_j} = \cos(\xi x_j) + i \sin(\xi x_j).$$

To understand stability we want to find $g(\xi)$.  To compute it, substitute form \eqref{ansatz} into scheme \eqref{ftcs}.  Indices ``$n+1$,'' ``$j-1$,'' and ``$j+1$'' will turn into powers.  Then use the properties of the exponential.  After simplification and trigonometric identities---please do the details in Exercise 1 below---we get
    $$g(\xi) = 1 - \frac{4Dk}{h^2} \sin^2\left(\frac{\xi h}{2}\right).$$

Absolute stability $|U_j^{n+1}| \le |U_j^{n}|$ corresponds to $|g(\xi)|\le 1$ for all $\xi \in \RR$.  For this scheme we get the condition $\displaystyle k \le \frac{h^2}{2D}$.  The same condition is derived via MOL in section 9.3.  In conclusion, the time step $k$ must be very small when the spacing $h$ is small.

\bigskip
\prob{Exercise 1. FTCS on heat equation}  Label the stencil, and then fill in the above details.

\bigskip
\hfill \begin{tikzpicture}[scale=1.5]
\node (left) at  (0.0,0.0) {};
\node (old) at   (1.0,0.0) {};
\node (right) at (2.0,0.0) {};
\node (new) at   (1.0,1.0) {};

\filldraw (left) circle (2.0pt);
\filldraw (old) circle (2.0pt);
\filldraw (right) circle (2.0pt);
\draw (new) circle (2.0pt);

\draw[line width=1.0pt] (left) -- (old) -- (right);
\draw[line width=1.0pt] (old) -- (new);
\end{tikzpicture}

\vfill


\clearpage
\newpage
\prob{Exercise 2. Crank-Nicolson on heat equation}  State the scheme.  von Neumann it.

\bigskip
\hfill \begin{tikzpicture}[scale=1.5]
\node (left) at  (0.0,0.0) {};
\node (old) at   (1.0,0.0) {};
\node (right) at (2.0,0.0) {};
\node (lnew) at  (0.0,1.0) {};
\node (new) at   (1.0,1.0) {};
\node (rnew) at  (2.0,1.0) {};

\filldraw (left) circle (2.0pt);
\filldraw (old) circle (2.0pt);
\filldraw (right) circle (2.0pt);
\draw (lnew) circle (2.0pt);
\draw (new) circle (2.0pt);
\draw (rnew) circle (2.0pt);

\draw[line width=1.0pt] (left) -- (old) -- (right);
\draw[line width=1.0pt] (old) -- (new);
\draw[line width=1.0pt] (lnew) -- (new) -- (rnew);
\end{tikzpicture}
\vfill

\prob{Exercise 3. Forward time and upwinding on the advection equation $u_t + a u_x = 0$, $a>0$ constant}  State the scheme, draw and label a stencil, and do the analysis.
\vfill

\prob{Exercise 4. CTCS (leapfrog) on the same advection equation}  Same instructions.
\vfill





\end{document}
