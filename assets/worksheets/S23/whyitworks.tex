\documentclass[11pt]{amsart}
%prepared in AMSLaTeX, under LaTeX2e
\addtolength{\oddsidemargin}{-.9in} 
\addtolength{\evensidemargin}{-.9in}
\addtolength{\topmargin}{-.9in}
\addtolength{\textwidth}{1.7in}
\addtolength{\textheight}{1.5in}

\renewcommand{\baselinestretch}{1.05}

\usepackage{verbatim} % for "comment" environment

\usepackage{palatino}

\usepackage[final]{graphicx}

\usepackage{tikz}
\usetikzlibrary{positioning}

\usepackage{enumitem,xspace,fancyvrb}

\newtheorem*{thm}{Theorem}
\newtheorem*{defn}{Definition}
\newtheorem*{example}{Example}
\newtheorem*{problem}{Problem}
\newtheorem*{remark}{Remark}

\DefineVerbatimEnvironment{mVerb}{Verbatim}{numbersep=2mm,frame=lines,framerule=0.1mm,framesep=2mm,xleftmargin=4mm,fontsize=\footnotesize}

% macros
\usepackage{amssymb}
\newcommand{\bA}{\mathbf{A}}
\newcommand{\bB}{\mathbf{B}}
\newcommand{\bE}{\mathbf{E}}
\newcommand{\bF}{\mathbf{F}}
\newcommand{\bJ}{\mathbf{J}}

\newcommand{\bb}{\mathbf{b}}
\newcommand{\bi}{\mathbf{i}}
\newcommand{\bj}{\mathbf{j}}
\newcommand{\bk}{\mathbf{k}}
\newcommand{\br}{\mathbf{r}}
\newcommand{\bu}{\mathbf{u}}
\newcommand{\bv}{\mathbf{v}}
\newcommand{\bw}{\mathbf{w}}
\newcommand{\bx}{\mathbf{x}}

\newcommand{\ppr}[1]{\frac{\partial #1}{\partial r}}
\newcommand{\ppt}[1]{\frac{\partial #1}{\partial t}}
\newcommand{\ppx}[1]{\frac{\partial #1}{\partial x}}
\newcommand{\ppy}[1]{\frac{\partial #1}{\partial y}}
\newcommand{\ppz}[1]{\frac{\partial #1}{\partial z}}

\newcommand{\Div}{\ensuremath{\nabla\cdot}}
\newcommand{\Curl}{\ensuremath{\nabla\times}}

\newcommand{\eps}{\epsilon}
\newcommand{\grad}{\nabla}
\newcommand{\ip}[2]{\ensuremath{\left<#1,#2\right>}}
\newcommand{\lam}{\lambda}
\newcommand{\lap}{\triangle}

\newcommand{\RR}{\mathbb{R}}
\newcommand{\ZZ}{\mathbb{Z}}
\newcommand{\prob}[1]{\bigskip\noindent\textbf{#1.}\quad }

\newcommand{\Matlab}{\textsc{Matlab}\xspace}

\newcommand{\ds}{\displaystyle}

\begin{document}
\scriptsize \noindent Math 615 NADE (Bueler) \hfill 24 February 2023 (revised) \quad \fbox{\emph{Not turned in!}}
\normalsize\medskip

\Large\centerline{\textbf{Summary: Why Finite Difference Methods Work}}
\medskip
\normalsize

\thispagestyle{empty}
\begin{quote}
\emph{Chapter 2 of the textbook (R.~J.~LeVeque, 2007) starts with the ODE BVP example}
	$$u''(x)=f(x), \qquad u(0)=\alpha, \quad u(1) = \beta.$$
\emph{It constructs a practical finite difference (FD) numerical method on this example.  Then it explains why the numerical solution will converge to the exact solution as we refine the grid ($m\to\infty$ and $h\to 0$).  We will use the same basic ``consistency $+$ stability $\implies$ convergence'' strategy on all problems, the \emph{Lax equivalence theorem}.  This summary puts the whole strategy on one page, for linear DEs, with details suppressed.}

\medskip
\noindent To use this as a worksheet: \emph{Fill in the extra space, or the reverse side, with your details!}
\end{quote}

\medskip

\prob{Stage 1.  Apply scheme to DE}  Choose the grid/mesh, including number of unknowns $m$ and spacing $h$.  Apply your FD \emph{discretization} or \emph{scheme}; it creates a \emph{family} of matrices $\{A^h\}$:
   $$\begin{pmatrix} \text{differential equation (DE)} \\ \text{and boundary/initial conditions} \end{pmatrix} \qquad \implies \qquad A^h U^h = F^h$$

\vfill
\prob{Stage 2.  Solve the scheme}  Numerical solution of the system of (linear) algebraic equations:
   $$A^h U^h = F^h \qquad \implies \qquad U^h = (A^h)^{-1} F^h$$

\vfill
\prob{Stage 3.  LTE and error equation}  Let $\hat U^h_j = u(x_j)$ be the grid values of the exact solution $u(x)$ of your DE.  (You may not know $u(x)$!)  Define the \emph{local truncation error} (LTE) as the residual from the scheme, when it is applied to the exact solution:
   $$\tau^h = A^h \hat U^h - F^h = O(h^p)$$
Taylor's theorem generates the \emph{order of accuracy} $p$, and if $p>0$ then the scheme is \emph{consistent}.  Defining the \emph{numerical error} $E^h = U^h - \hat U^h$, get:
   $$A^h E^h = -\tau^h \qquad \implies \qquad E^h = - (A^h)^{-1} \tau^h$$

\vfill
\prob{Stage 4.  Apply stability to show convergence}  Show \emph{stability}: there is $C>0$ so that $\|(A^h)^{-1}\| \le C$ for all $h>0$.  (Stability may be difficult to show!)  Because $\|\tau^h\| = O(h^p)$, get \emph{convergence} at rate $p$:
	$$\|E^h\| = \|- (A^h)^{-1} \tau^h\| \le \|(A^h)^{-1}\| \|\tau^h\| \le C O(h^p) = O(h^p)$$

\vfill
\end{document}
