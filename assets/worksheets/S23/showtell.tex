\documentclass[12pt]{amsart}
%prepared in AMSLaTeX, under LaTeX2e
\addtolength{\oddsidemargin}{-.85in}
\addtolength{\evensidemargin}{-.85in}
\addtolength{\topmargin}{-0.85in}
\addtolength{\textwidth}{1.5in}
\addtolength{\textheight}{1.6in}
\newcommand{\normalspacing}{\renewcommand{\baselinestretch}{1.05}
        \tiny\normalsize}

\newtheorem*{thm}{Theorem}
\newtheorem*{defn}{Definition}
\newtheorem*{example}{Example}
\newtheorem*{problem}{Problem}
\newtheorem*{remark}{Remark}

\usepackage{amssymb,fancyvrb,xspace,bm}
\usepackage{palatino}

\usepackage[final]{graphicx}

\usepackage[pdftex, colorlinks=true, plainpages=false, linkcolor=black, citecolor=red, urlcolor=red]{hyperref}

% macros
\newcommand{\ba}{\mathbf{a}}
\newcommand{\bb}{\mathbf{b}}
\newcommand{\bbf}{\mathbf{f}}
\newcommand{\bn}{\mathbf{n}}
\newcommand{\br}{\mathbf{r}}
\newcommand{\bu}{\mathbf{u}}
\newcommand{\bv}{\mathbf{v}}
\newcommand{\bx}{\mathbf{x}}
\newcommand{\by}{\mathbf{y}}

\newcommand{\bT}{\mathbf{T}}

\newcommand{\CC}{\mathbb{C}}
\newcommand{\Div}{\nabla\cdot}
\newcommand{\eps}{\epsilon}
\newcommand{\grad}{\nabla}
\newcommand{\ZZ}{\mathbb{Z}}
\newcommand{\ip}[2]{\ensuremath{\left<#1,#2\right>}}
\newcommand{\lam}{\lambda}
\newcommand{\lap}{\triangle}
\newcommand{\RR}{\mathbb{R}}

\newcommand{\prob}[1]{\bigskip\noindent\large\textbf{#1}.\,\normalsize }
\newcommand{\ppart}[1]{\textbf{(#1)}\,\, }
\newcommand{\epart}[1]{\medskip\noindent\textbf{(#1)}\,\, }

\newcommand{\pts}[1]{\scriptsize [#1 points] \normalsize}

\newcommand{\Matlab}{\textsc{Matlab}\xspace}
\newcommand{\Octave}{\textsc{Octave}\xspace}
\newcommand{\Python}{\textsc{Python}\xspace}


\begin{document}
\scriptsize% \phantom{bob} \vspace{-0.3in}
\noindent Math 615 NADE \, (Bueler) \hfill  \today
\normalsize\bigskip
\normalspacing

\Large\centerline{\textbf{Show and tell with \textbf{PETSc} and \textbf{Firedrake}}}
\normalsize

\bigskip
\thispagestyle{empty}
\normalspacing

\renewcommand{\labelenumi}{\arabic{enumi}.\,}


\bigskip
\noindent \textbf{Examples from my book.}  I wrote a book called \emph{PETSc for Partial Differential Equations} which was published by SIAM Press in 2021.  The C and Python codes for the book's examples are at \,\href{https://github.com/bueler/p4pdes}{\texttt{github.com/bueler/p4pdes}}.  In this demonstration I'll show examples from Chapters 5, 11, and 14.

\medskip
\noindent \textbf{What is PETSc?}  The \emph{Portable, Extensible Toolkit for Scientific computing} is an open and free C library of numerical software, especially linear algebra, mesh management, and ODE IVP solvers, from Argonne National Laboratory.  Starting in about 1990, PETSc co-evolved with MPI (= \emph{Message Passing Interface}), also from Argonne, as the fundamental infrastructure for doing science and engineering simulations and computations on supercomputers, the largest of which have more than a million processors (cores).  MPI and PETSc are essential ``software stack'' for most large-scale \emph{parallel} computations.

Documentation and download is at \,\href{https://petsc.org/}{\texttt{petsc.org}}.

\bigskip
\noindent \textbf{Example 1.}  Chapter 5 solves a pair of \emph{coupled diffusion-reaction equations} on $(x,y) \in (0,2.5)\times (0,2.5)$ and $t>0$:
\begin{align*}
u_t &= D_u \grad^2 u - u v^2 + \phi (1 - u) \\
v_t &= D_v \grad^2 v + u v^2 - (\phi + \kappa) v
\end{align*}
where $D_u,D_v,\phi,\kappa$ are constants and $u(t,x,y),v(t,x,y)$ are chemical concentrations.  This is a model for pattern generation, for instance as an explanation of how animal skins can end up spotted.

The C code \texttt{pattern.c} calls the PETSc library for time-stepping, parallel grid management, and parallel, iterative solvers for linear systems.  The spatial derivatives are approximated with a 9-point-stencil centered finite difference scheme for $\nabla^2$, which generates an MOL system.  Default time-stepping (\texttt{ts\_}) is by an adaptive method which is implicit for the stiff diffusion part and explicit for the non-stiff, nonlinear reaction terms.  Other methods can be chosen at run-time, for instance by \texttt{-ts\_type beuler}, and so on.

Here is how to build it, and run it in parallel (4 cores) on a $96\times 96$ spatial grid:

\medskip
\begin{Verbatim}[fontsize=\small]
$ cd c/ch5/ && make pattern
$ mpiexec -n 4 ./pattern -da_refine 5 -ts_max_time 5000 -ts_monitor \
    -ts_monitor_solution draw
\end{Verbatim}

\bigskip
\noindent \textbf{Example 2.}  The \texttt{advect.c} code in Chapter 11 takes a finite volume approach, to generate the MOL ODE system, for solving a scalar \emph{advection equation} in 2D, on $(x,y) \in (-1,1)\times (-1,1)$, with periodic boundary conditions, for $t>0$:
    $$u_t + \nabla \cdot (\mathbf{a} u) = 0$$
The velocity field in this example is rotational: $\mathbf{a}(x,y) = \left<y, -x\right>$.

Here is an example run (4 cores, $160\times 160$ spatial grid):

\medskip
\begin{Verbatim}[fontsize=\small]
$ cd c/ch11/ && make advect
$ mpiexec -n 4 ./advect -da_refine 5 -adv_problem rotation \
    -ts_max_time 6.283185 -ts_monitor -ts_monitor_solution draw
\end{Verbatim}

\medskip
A surface plot of the initial condition $u(x,y,0)$ would look like a cone and a square tower.  The spatial derivatives are approximated with finite differences and a flux-limited higher-order upwind scheme.  The time-stepping is by a 3rd-order adaptive Runge-Kutta method, quite suitable for such hyperbolic problems if the fluxes are discretized appropriately.  The time-dependent solution rotates the initial picture.  We see that numerical diffusion causes the sharp edges to smooth out.  

\bigskip
\noindent \textbf{Example 3.}  The last example is the Python code in Chapter 14 which solves a Stokes problem for a steady flow of a 2D viscous, incompressible fluid with velocity $\bu$, pressure $p$, and viscosity $\mu$.  The equations are
\begin{align*}
- \Div \left(2 \mu\, D \bu\right) + \grad p &= \bm{0}, \\
\Div \bu &= 0.
\end{align*}
The boundary value problem has zero velocity $\bu=0$ on the bottom and sides of the unit square, but along the top we impose right-ward motion; this is called a \emph{lid-driven cavity}.

\emph{Firedrake} (\href{https://www.firedrakeproject.org/}{\texttt{firedrakeproject.org}}) is a Python finite element library which allows us to express a PDE problem directly, independent of a choice of particular (spatial) discretization.

\medskip
\begin{Verbatim}[fontsize=\small]
V = VectorFunctionSpace(mesh, 'CG', degree=2)
W = FunctionSpace(mesh, 'CG', degree=1)
Z = V * W
up = Function(Z)
u,p = split(up)
v,q = TestFunctions(Z)
Du = 0.5 * (grad(u) + grad(u).T)
Dv = 0.5 * (grad(v) + grad(v).T)
F = (2.0 * args.mu * inner(Du,Dv) - p * div(v) - div(u) * q) * dx
\end{Verbatim}

FIXME

\end{document}
